%# -*- coding:utf-8 -*-
%%%%%%%%%%%%%%%%%%%%%%%%%%%%%%%%%%%%%%%%%%%%%%%%%%%%%%%%%%%%%%%%%%%%%%%%%%%%%%%%%%%%%


\chapter{西门庆露阳惊爱月\KG 李瓶儿睹物哭官哥}


诗曰:

\[
枫叶初丹槲叶黄,河阳愁鬓恰新霜。
鬼门徒忆空回首,泉路凭谁说断肠?
路杳云迷愁漠漠,珠沉玉殒事茫茫。
惟有泪珠能结雨,尽倾东海恨无疆。
\]

话说孟玉楼和潘金莲,在门首打发磨镜叟去了。忽见从东一人,带着大帽眼纱骑着骡子,走得甚急,迳到门首下来,慌的两个妇人往后走不迭。落后揭开眼纱却是韩伙计来家了。平安忙问道:“货车到了不曾?”韩道国道:“货车进城了禀问老爹卸在那里?”平安道:“爹不在家,往周爷府里吃酒去了,教卸在对门楼上哩。你老人家请进里边去。”不一时,陈敬济出来,陪韩道国入后边见了月娘出来厅上,拂去尘土,把行李搭裢教王经送到家去。月娘一面打发出饭来与他吃了。不一时,货车才到。敬济拿钥匙开了那边楼上门,就有卸车的小脚子领筹搬运一箱箱都堆卸在楼上。十大车缎货,直卸到掌灯时分。崔本也来帮扶。完毕,查数锁门,贴上封皮,打发小脚钱出门。早有玳安往守备府报西门庆去了。

西门庆听见家中卸货,吃了几杯酒,约掌灯以后就来家。韩伙计等着见了,在厅上坐的,悉把前后往回事说了一遍。西门庆因问:“钱老爹书下了,也见些分上不曾?”韩道国道:“全是钱老爹这封书,十车货少使了许多税钱。小人把段箱,两箱并一箱,三停只报了两停,都当茶叶、马牙香柜上税过来了。通共十大车货,只纳了三十两五钱钞银子。老爹接了报单,也没差巡拦下来查点,就把车喝过来了。”西门庆听言,满心欢喜,因说:“到明日,少不的重重买一分礼谢他。”于是吩咐陈敬济陪韩伙计、崔大哥坐,后边拿菜出来,留吃了一回酒,方才各散回家。

王六儿听见韩道国来了,吩咐丫头春香、锦儿,伺候下好茶好饭。等的晚上,韩道国到家,拜了家堂,脱了衣裳,净了面目,夫妻二人各诉离情一遍。韩道国悉把买卖得意一节告诉老婆,老婆又见搭裢内沉沉重重许多银两,因问他,替己又带了一二百两货物酒米,卸在门外店里,慢慢发卖了银子来家。老婆满心欢喜道:“我听见王经说,又寻了个甘伙计做卖手,咱每和崔大哥与他同分利钱使,这个又好了。到出月开铺了。”韩道国道:“这里使着了人做卖手,南边还少个人立庄置货老爹一定还裁派我去。”老婆道:“你看货才料,自古能者多劳。你不会做买卖那老爹托你么!常言:不将辛苦意,难得世间财。你外边走上三年,你若懒得去等我对老爹说了,教姓甘的和保官儿打外,你便在家卖货就是了。”韩道国道:“外边走熟了,也罢了。”老婆道:“可又来,你先生迷了路,在家也是闲!”说毕,摆上酒来,夫妇二人饮了几杯阔别之酒,收拾就寝。是夜欢娱无度,不必细说。次日却是八月初一日,韩道国早到房子内,同崔本、甘伙计看着收拾装修土库,不在话下。

却说西门庆见货物卸了,家中无事,忽然心中想起要往郑爱月儿家去。暗暗使玳安儿送了三两银子、一套纱衣服与他。郑家鸨子听见西门老爹来请他家姐儿,如天上落下来的一般,连忙收下礼物,没口子向玳安道:“你多顶上老爹,就说他姐儿两个都在家里伺候老爹,请老爹早些儿下降。”玳安走来家中书房内,回了西门庆话。西门庆约午后时分,吩咐玳安收拾着凉轿,头上戴着披巾,身上穿青纬罗暗补子直身,粉底皂靴,先走在房子看了一回装修土库,然后起身,坐上凉轿,放下斑竹帘来,琴童、玳安跟随,留王经在家,止叫春鸿背着直袋,迳往院中郑爱月儿家。正是:

\[
天仙机上整香罗,入手先拖雪一窝。
不独桃源能问渡,却来月窟伴嫦娥。
\]

却说郑爱香儿打扮的粉面油头,见西门庆到,笑吟吟在半门里首迎接进去。到于明间客位,道了万福。西门庆坐下,就吩咐小厮琴童:“把轿回了家去,晚夕骑马来接。”琴童跟轿家去,止留玳安和春鸿两个伺候。少顷,鸨子出来拜见,说道“外日姐儿在宅内多有打搅,老爹来这里,自恁走走罢了,如何又赐将礼来?又多谢与姐儿的衣服。”西门庆道:“我那日叫他,怎的不去?——只认王皇亲家了!”鸨子道:“俺每如今还怪董娇儿和李桂儿。不知是老爹生日叫唱,他每都有了礼,只俺们姐儿没有。若早知时,决不答应王皇亲家唱,先往老爹宅里去了。落后,老爹那里又差了人来,慌的老身背着王家人,连忙撺掇姐儿打后门上轿去了。”西门庆道:“先日我在他夏老爹家酒席上,就定下他了。他若那日不去,我不消说的就恼了。怎的他那日不言不语,不做喜欢,端的是怎么说?”鸨子道:“小行货子家,自从梳弄了,那里好生出去供唱去!到老爹宅内,见人多,不知唬的怎样的。他从小是恁不出语,娇养惯了。你看,甚时候才起来!老身该催促了几遍,说老爹今日来,你早些起来收拾了罢。他不依,还睡到这咱晚。”

不一时,丫鬟拿茶上来,郑爱香儿向前递了茶吃了。鸨子道:“请老爹到后边坐罢。”郑爱香儿就让西门庆进入郑爱月儿的房外明间内坐下,西门庆看见上面楷书“爱月轩”三字。坐了半日,忽听帘栊响处,郑爱月儿出来,不戴\textuni{4BFC}髻,头上挽着一窝丝杭州缵,梳的黑\textuni{29B79}\textuni{29B79}光油油的乌云,云鬓堆鸦犹若轻烟密雾。上着白藕丝对衿仙裳,下穿紫绡翠纹裙,脚下露红鸳凤嘴鞋,前摇宝玉玲珑,越显那芙蓉粉面。正是:

\[
若非道子观音画,定然延寿美人图。
\]

爱月儿走到下面,望上不端不正与西门庆道了万福,就用洒金扇儿掩着粉脸坐在旁边。西门庆注目停视,比初见时节越发齐整,不觉心摇目荡,不能禁止。不一时,丫鬟又拿一道茶来。这粉头轻摇罗袖,微露春纤,取一钟,双手递与西门庆,然后与爱香各取一钟相陪。吃毕,收下盏托去,请宽衣服房里坐。西门庆叫玳安上来,把上盖青纱衣宽了,搭在椅子上。进入粉头房中,但见瑶窗绣幕,锦褥华裀,异香袭人,极其清雅,真所谓神仙洞府,人迹不可到者也。彼此攀话调笑之际,只见丫鬟进来安放桌儿,摆下许多精制菜蔬。先请吃荷花细饼,郑爱月儿亲手拣攒肉丝,卷就,安放小泥金碟儿内,递与西门庆吃。须臾,吃了饼,收了家火去,就铺茜红毡条,取出牙牌三十二扇,与西门庆抹牌。抹了一回,收过去,摆上酒来。但见盘堆异果,酒泛金波,十分齐整。姊妹二人递了酒,在旁筝排雁柱,款跨绞绡——爱香儿弹筝,爱月儿琵琶,唱了一套“兜的上心来”。端的词出佳人口,有裂石绕梁之声。唱毕,促席而坐,拿骰盆儿与西门庆抢红猜枚。

饮够多时,郑爱香儿推更衣出去了,独有爱月儿陪着西门庆吃酒。先是西门庆向袖中取出白绫汗巾儿,上头束着个金穿心盒儿。郑爱月儿只道是香茶,便要打开西门庆道:“不是香茶,是我逐日吃的补药。我的香茶不放在这里面,只用纸包着。”于是袖中取出一包香茶桂花饼儿递与他。那爱月儿不信,还伸手往他袖子里掏,又掏出个紫绉纱汗巾儿,上拴着一副拣金挑牙儿,拿在手中观看,甚是可爱。说道:“我见桂姐和吴银姐都拿着这样汗巾儿,原来是你与他的。”西门庆道:“是我扬州船上带来的。不是我与他,谁与他的?你若爱,与了你罢。到明日,再送一副与你姐姐。”说毕,西门庆就着钟儿里酒,把穿心盒儿内药吃了一服,把粉头搂在怀中,两个一递一口儿饮酒咂舌,无所不至。西门庆又舒手摸弄他香乳,紧紧就就赛麻圆滑腻。一面扯开衫儿观看,白馥馥犹如莹玉一般。揣摩良久,淫心辄起,腰间那话突然而兴。解开裤带,令他纤手笼攥。粉头见其粗大,唬的吐舌害怕,双手搂定西门庆脖项说道:“我的亲亲,你今日初会,将就我,只放半截儿罢!若都放进去,我就死了。你敢吃药养的这等大,不然,如何天生恁怪剌剌儿的——红赤赤,紫漒漒,好砢碜人子!”西门庆笑道:“我的儿!你下去替我品品。”爱月儿道:“慌怎的,往后日子多如树叶儿。今日初会,人生面不熟,再来等我替你品。”说毕,西门庆欲与他交欢,爱月儿道:“你不吃酒了?”西门庆道:“我不吃了,咱睡罢。”爱月儿便叫丫鬟把酒桌抬过一边,与西门庆脱靴,他便往后边更衣澡牝去了。西门庆脱靴时,还赏了丫头一块银子,打发先上床睡,炷了香,放在薰笼内。良久,妇人进房,问西门庆:“你吃茶不吃?”西门庆道:“我不吃。”一面掩上房门,放下绫绡来,将绢儿安放在褥下,解衣上床。两个枕上鸳鸯,被中鸂鶒。西门庆见粉头肌肤纤细,牝净无毛,犹如白面蒸饼一般,柔嫩可爱。抱了抱腰肢,未盈一掬。诚为软玉温香,千金难买。于是把他两只白生生银条般嫩腿儿夹在两边腰眼间,那话上使了托子,向花心里顶入。龟头昂大,濡搅半晌,方才没棱。那爱月儿把眉头绉在一处,两手攀搁在枕上,隐忍难挨。朦胧着星眼,低声说道:“今日你饶了郑月儿罢!”西门庆听了,愈觉销魂,肆行抽送,不胜欢娱。正是:得多少——

\[
春点桃花红绽蕊,风欺杨柳绿翻腰。
\]

西门庆与郑月儿留恋至三更方才回家。到次日,吴月娘打发他往衙门中去了,和玉楼、金莲、李娇儿都在上房坐的。只见玳安进来上房取尺头匣儿,往夏提刑送生日礼去。月娘因问玳安:“你爹昨日坐轿于往谁家吃酒,吃到那咱晚才回家?想必又在韩道国家,望他那老婆去来。原来贼囚根子成日只瞒着我,背地替他干这等茧儿!”玳安道:“不是。他汉子来家,爹怎好去的!”月娘道:“不是那里,却是谁家?”那玳安又不说,只是笑。取了段匣,送礼去了。潘金莲道:“大姐姐,你问这贼囚根子,他怎肯实说?我听见说蛮小厮昨日也跟了去来,只叫蛮小厮来问就是了。”一面把春鸿叫到跟前。金莲问:“你昨日跟了你爹轿子去,在谁家吃酒来?你实说便罢,不实说,如今你大娘就要打你。”那春鸿跪下便道:“娘休打小的,待小的说就是了。小的和玳安、琴童哥三个,跟俺爹从一座大门楼进去,转了几条街巷,到个人家,只半截门儿,都用锯齿儿镶了。门里立着个娘娘,打扮的花花黎黎的。”金莲听见笑了,说道:“囚根子,一个院里半门子也不认的?赶着粉头叫娘娘起来。”又问道:“那个娘娘怎么模样?你认的他不认的?”春鸿道:“我不认的他,也象娘每头上戴着这个假壳。进入里面,一个白头的阿婆出来,望俺爹拜了一拜。落后请到后边,又是一位年小娘娘出来,不戴假壳,生的瓜子面,搽的嘴唇红红的,陪着俺爹吃酒。”金莲道:“你们都在那里坐来?”春鸿道:“我和玳安、琴童哥便在阿婆房里,陪着俺每吃酒并肉兜子来。”把月娘、玉楼笑的了不得。因问道:“你认的他不认的?”春鸿道:“那一个好似在咱家唱的。”玉楼笑道:“就是李桂姐了。”月娘道:“原来摸到他家去来。”李娇儿道:“俺家没半门子。”金莲道:“只怕你家新安了半门子是的。”问了一回。西门庆来家,就往夏提刑家拜寿去了。

却说潘金莲房中养的一只白狮子猫儿,浑身纯白,只额儿上带龟背一道黑,名唤雪里送炭,又名雪狮子。又善会口衔汗巾子,拾扇儿。西门庆不在房中,妇人晚夕常抱他在被窝里睡,又不撒尿屎在衣服上,呼之即至,挥之即去,妇人常唤他是雪贼。每日不吃牛肝干鱼,只吃生肉,调养的十分肥壮,毛内可藏一鸡蛋。甚是爱惜他,终日在房里用红绢裹肉,令猫扑而挝食。这日也是合当有事,官哥儿心中不自在,连日吃刘婆子药,略觉好些。李瓶儿与他穿上红缎衫儿,安顿在外间炕上顽耍,迎春守着,奶子便在旁吃饭。不料这雪狮子正蹲在护炕上,看见官哥儿在炕上,穿着红衫儿一动动的顽耍,只当平日哄喂他肉食一般,猛然望下一跳,将官哥儿身上皆抓破了。只听那官哥儿“呱”的一声,倒咽了一口气,就不言语了,手脚俱风搐起来。慌的奶子丢下饭碗,搂抱在怀,只顾唾哕与他收惊。那猫还来赶着他要挝,被迎春打出外边去了。如意儿实承望孩子搐过一阵好了,谁想只顾常连,一阵不了一阵搐起来。忙使迎春后边请李瓶儿去,说:“哥儿不好了,风搐着哩,娘快去!”那李瓶儿不听便罢,听了,正是:

\[
惊损六叶连肝肺,唬坏三毛七孔心。
\]

连月娘慌的两步做一步,迳扑到房中。见孩子搐的两只眼直往上吊,通不见黑眼睛珠儿,口中白沫流出,咿咿犹如小鸡叫,手足皆动。一见心中犹如刀割相侵,连忙搂抱起来,脸揾着他嘴儿,大哭道:“我的哥哥,我出去好好儿,怎么就搐起来?”迎春与奶子,悉把被五娘房里猫所唬一节说了。那李瓶儿越发哭起来,说道:“我的哥哥,你紧不可公婆意,今日你只当脱不了打这条路儿去了!”月娘听了,一声儿没言语,一面叫将金莲来,问他说:“是你屋里的猫唬了孩子?”金莲问:“是谁说的?”月娘指着:“是奶子和迎春说来。”金莲道:“你看这老婆子这等张嘴!俺猫在屋里好好儿的卧着不是。你每怎的把孩子唬了,没的赖人起来。爪儿只拣软处捏,俺每这屋里是好缠的!”月娘道:“他的猫怎得来这屋里?”迎春道:“每常也来这边屋里走跳。”金莲接过来道:“早时你说,每常怎的不挝他?可可今日儿就挝起来?你这丫头也跟着他恁张眉瞪眼儿,六说白道的。将就些儿罢了,怎的要把弓儿扯满了?可可儿俺每自恁没时运来。”于是使性子抽身往房里去了。看官听说:潘金莲见李瓶儿有了官哥儿,西门庆百依百随,要一奉十,故行此阴谋之事,驯养此猫,必欲唬死其子,使李瓶儿宠衰,教西门庆复亲于己。就如昔日屠岸贾养神獒害赵盾丞相一般。正是:

\[
花枝叶底犹藏刺,人心怎保不怀毒。
\]

月娘众人见孩子只顾搐起来,一面熬姜汤灌他,一面使来安儿快叫刘婆去。不一时,刘婆子来到,看了脉息,只顾跌脚,说道:“此遭惊唬重了,难得过了。快熬灯心薄荷金银汤。”取出一丸金箔丸来,向钟儿内研化。牙关紧闭,月娘连忙拔下金簪儿来,撬开口,灌下去。刘婆道:“过得来便罢。如过不来,告过主家奶奶,必须要灸几醮才好。”月娘道:“谁敢耽?必须等他爹来问了不敢。灸了,惹他来家吆喝。”李瓶儿道:“大娘救他命罢!若等来家,只恐迟了。若是他爹骂,等我承当就是了。”月娘道:“孩儿是你的孩儿,随你灸,我不敢张主,”当下,刘婆子把官哥儿眉攒、脖根、两手关尺并心口,共灸了五醮,放他睡下。那孩子昏昏沉沉,直睡到日暮时分西门庆来家还不醒。那刘婆见西门庆来家,月娘与了他五钱银子,一溜烟从夹道内出去了。

西门庆归到上房,月娘把孩子风搐不好对西门庆说了,西门庆连忙走到前边来看视,见李瓶儿哭的眼红红的,问:“孩儿怎的风搐起来?”李瓶儿满眼落泪,只是不言语。问丫头、奶子,都不敢说。西门庆又见官哥手上皮儿去了,灸的满身火艾,心中焦燥,又走到后边问月娘。月娘隐瞒不住,只得把金莲房中猫惊唬之事说了:“刘婆子刚才看,说是急惊风,若不针灸,难过得来。若等你来,只恐怕迟了。他娘母子自主张,叫他灸了孩儿身上五醮,才放下他睡了。这半日还未醒。”西门庆不听便罢,听了此言,三尸暴跳,五脏气冲,怒从心上起,恶向胆边生,直走到潘金莲房中,不由分说,寻着雪狮子,提着脚走向穿廊,望石台基轮起来只一摔,只听响亮一声,脑浆迸万朵桃花,满口牙零噙碎玉。正是:

\[
不在阳间擒鼠耗,却归阴府作狸仙。
\]

潘金莲见他拿出猫去摔死了,坐在炕上风纹也不动。待西门庆出了门,口里喃喃呐呐骂道:“贼作死的强盗,把人妆出去杀了才是好汉!一个猫儿碍着你噇屎?亡神也似走的来摔死了。他到阴司里,明日还问你要命,你慌怎的?贼不逢好死变心的强盗!”西门庆走到李瓶儿房里,因说奶子、迎春:“我教你好看着孩儿,怎的教猫唬了他,把他手也挝了!又信刘婆子那老淫妇,平白把孩子灸的恁样的。若好便罢,不好,把这老淫妇拿到衙门里,与他两拶!”李瓶儿道:“你看孩儿紧自不得命,你又是恁样的。孝顺是医家,他也巴不得要好哩。”李瓶儿只指望孩儿好来,不料被艾火把风气反于内,变为慢风,内里抽搐的肠肚儿皆动,尿屎皆出,大便屙出五花颜色,眼目忽睁忽闭,终朝只是昏沉不省,奶也不吃了。李瓶儿慌了,到处求神问卜打卦,皆有凶无吉。月娘瞒着西门庆又请刘婆子来家跳神,又请小儿科太医来看。都用接鼻散试之:若吹在鼻孔内打鼻涕,还看得;若无鼻涕出来,则看阴骘守他罢了。于是吹下去,茫然无知,并无一个喷涕出来。越发昼夜守着哭涕不止,连饮食都减了。

看看到八月十五日将近,月娘因他不好,连自家生日都回了不做,亲戚内眷,就送礼来也不请。家中止有吴大妗子、杨姑娘并大师父来相伴。那薛姑子和王姑子两个,在印经处争分钱不平,又使性儿,彼此互相揭调。十四日,贲四同薛姑子催讨,将经卷挑将米,一千五百卷都完了。李瓶儿又与了一吊钱买纸马香烛。十五日同陈敬济早往岳庙里进香纸,把经看着都散施尽了,走来回李瓶儿话。乔大户家,一日一遍使孔嫂儿来看,又举荐了一个看小儿的鲍太医来看,说道:“这个变成天吊客忤,治不得了。”白与了他五钱银子,打发去了。灌下药去也不受,还吐出了。只是把眼合着,口中咬的牙格支支响。李瓶儿通衣不解带,昼夜抱在怀中,眼泪不干的只是哭。西门庆也不往那里去,每日衙门中来家,就进来看孩儿。

那时正值八月下旬天气,李瓶儿守着官哥儿睡在床上,桌上点着银灯,丫鬟养娘都睡熟了。觑着满窗月色,更漏沉沉,果然愁肠万结,离思千端。正是:

\[
人逢喜事精神爽,闷来愁肠瞌睡多。
\]
但见:

\[
银河耿耿,玉漏迢迢。穿窗皓月耿寒光,透户凉风吹夜气。樵楼禁鼓,一更未尽一更敲;别院寒砧,千捣将残千捣起。画檐前叮当铁马,敲碎思妇情怀;银台上闪烁灯光,偏照佳人长叹。一心只想孩儿好,谁料愁来睡梦多。
\]
当下,李瓶儿卧在床上,似睡不睡,梦见花子虚从前门外来,身穿白衣,恰似活时一般。见了李瓶儿,厉声骂道:“泼贼淫妇,你如何抵盗我财物与西门庆?如今我告你去也。”被李瓶儿一手扯住他衣袖,央及道:“好哥哥,你饶恕我则个!”花子虚一顿,撒手惊觉,却是南柯一梦。醒来,手里扯着却是官哥儿的衣衫袖子。连哕了几口道:“怪哉!怪哉!”听一听更鼓,正打三更三点。李瓶儿唬的浑身冷汗,毛发皆竖。

到次日,西门庆进房来,就把梦中之事告诉一遍。西门庆道:“知道他死到那里去了!此是你梦想旧境。只把心来放正着,休要理他。如今我使小厮拿轿子接了吴银儿来,与你做个伴儿。再把老冯叫来伏侍两日。”玳安打院里接了吴银儿来。那消到日西时分,那官哥儿在奶子怀里只搐气儿了。慌的奶子叫李瓶儿:“娘,你来看哥哥,这黑眼睛珠儿只往上翻,口里气儿只有出来的,没有进去的。”这李瓶儿走来抱到怀中,一面哭起来,叫丫头:“快请你爹去!你说孩子待断气也。”可可常峙节又走来说话,告诉房子儿寻下了,门面两间,二层,大小四间,只要三十五两银子。西门庆听见后边官哥儿重了,就打发常峙节起身,说:“我不送你罢,改日我使人拿银子和你看去。”急急走到李瓶儿房中。月娘众人都在房里瞧着,那孩子在他娘怀里一口口搐气儿。西门庆不忍看他,走到明间椅子上坐着,只长吁短叹。那消半盏茶时,官哥儿呜呼哀哉,断气身亡。时八月廿三日申时也,只活了一年零两个月。合家大小放声号哭。那李瓶儿挝耳挠腮,一头撞在地下,哭的昏过去。半日方才苏省,搂着他大放声哭叫道:“我的没救星儿,心疼杀我了!宁可我同你一答儿里死了罢,我也不久活在世上了。我的抛闪杀人的心肝,撇的我好苦也!”那奶子如意儿和迎春在旁,哭的言不得,动不得。西门庆即令小厮收拾前厅西厢房干净,放下两条宽凳,要把孩子连枕席被褥抬出去那里挺放。那李瓶儿倘在孩儿身上,两手搂抱着,那里肯放!口口声声直叫:“没救星的冤家!娇娇的儿!生揭了我的心肝去了!撇的我枉费辛苦,干生受一场,再不得见你了,我的心肝!……”月娘众人哭了一回,在旁劝他不住。西门庆走来,见他把脸抓破了,滚的宝髻蓬松,乌云散乱,便道:“你看蛮的!他既然不是你我的儿女,干养活他一场,他短命死了,哭两声丢开罢了,如何只顾哭了去!又哭不活他,你的身子也要紧。如今抬出去,好叫小厮请阴阳来看。——这是甚么时候?”月娘道:“这个也有申时前后。”玉楼道:“我头里怎么说来?他管情还等他这个时候才去。——原是申时生,还是申时死。日子又相同,都是二十三日,只是月分差些。圆圆的一年零两个月。”李瓶儿见小厮每伺候两旁要抬他,又哭了,说道:“慌抬他出去怎么的?大妈妈,你伸手摸摸,他身上还热哩!”叫了一声:“我的儿嚛!你教我怎生割舍的你去?坑得我好苦也!……”一头又撞倒在地下,哭了一回。众小厮才把官哥儿抬出,停在西厢房内。

月娘向西门庆计较:“还对亲家那里并他师父庙里说声去。”西门庆道,“他师父庙里,明早去罢。”一面使玳安往乔大户家说了,一面使人请了徐阴阳来批书。又拿出十两银子与贲四,教他快抬了一付平头杉板,令匠人随即攒造了一具小棺椁儿,就要入殓。乔宅那里一闻来报,乔大户娘子随即坐轿子来,进门就哭。月娘众人又陪着大哭了一场,告诉前事一遍。不一时,阴阳徐先生来到,看了,说道:“哥儿还是正申时永逝。”月娘吩咐出来,教与他看看黑书。徐先生将阴阳秘书瞧了一回,说道:“哥儿生于政和丙申六月廿三日申时,卒于政和丁酉八月廿三日申时。月令丁酉,日干壬子,犯天地重丧,本家要忌:忌哭声。亲人不忌。入殓之时,蛇、龙、鼠、兔四生人,避之则吉。又黑书上云:壬子日死者,上应宝瓶宫,下临齐地。他前生曾在兖州蔡家作男子,曾倚力夺人财物,吃酒落魄,不敬天地六亲,横事牵连,遭气寒之疾,久卧床席,秽污而亡。今生为小儿,亦患风痫之疾。十日前被六畜惊去魂魄,又犯土司太岁,先亡摄去魂魄,托生往郑州王家为男子,后作千户,寿六十八岁而终。”须臾,徐先生看了黑书,请问老爹,明日出去或埋或化,西门庆道:“明日如何出得!搁三日,念了经,到五日出去,坟上埋了罢。”徐先生道:“二十七日丙辰,合家本命都不犯,宜正午时掩土。”批毕书,一面就收拾入殓,已有三更天气。李瓶儿哭着往房中,寻出他几件小道衣、道髻、鞋袜之类,替他安放在棺椁内,钉了长命钉,合家大小又哭了一场,打发阴阳去了。

次日,西门庆乱着,也没往衙门中去。夏提刑打听得知,早晨衙门散时,就来吊问。又差人对吴道官庙里说知,到三日,请报恩寺八众僧人在家诵经。吴道官庙里并乔大户家,俱备折卓三牲来祭奠。吴大舅、沈姨夫、门外韩姨夫、花大舅都有三牲祭卓来烧纸。应伯爵、谢希大、温秀才、常峙节、韩道国、甘出身、贲第传、李智、黄四都斗了分资,晚夕来与西门庆伴宿。打发僧人去了,叫了一起提偶的,先在哥儿灵前祭毕,然后,西门庆在大厅上放桌席管待众人。那日院中李桂姐、吴银儿并郑月儿三家,都有人情来上纸。

李瓶儿思想官哥儿,每日黄恹恹,连茶饭儿都懒待吃,题起来只是哭涕,把喉音都哭哑了。西门庆怕他思想孩儿,寻了拙智,白日里吩咐奶子、丫鬟和吴银儿相伴他,不离左右。晚夕,西门庆一连在他房中歇了三夜,枕上百般解劝。薛姑子夜间又替他念《楞严经》、《解冤咒》,劝他:“休要哭了。他不是你的儿女,都是宿世冤家债主。《陀罗经》上不说的好:昔日有一妇人,生产孩儿三遍,俱不过两岁而亡,妇人悲啼不已。抱儿江边,不忍抛弃。感得观世音菩萨化作一僧,谓此妇人曰:‘不用啼哭,此非你儿,是你生前冤家。三度托生,皆欲杀汝。你若不信,我交你看。’将手一指,其儿遂化作一夜叉之形,向水中而立,报言:‘汝曾杀我来,我特来报冤。今因汝常持《佛顶心陀罗经》,善神日夜拥护,所以杀汝个得。我已蒙观世音菩萨受度了,从今永不与汝为冤。’道毕,遂沉水中不见。不该我贫僧说,你这儿子,必是宿世冤家,托来你荫下,化目化财,要恼害你身。为你舍了此《佛顶心陀罗经》一千五百卷,有此功行,他害你不得,故此离身。到明日再生下来,才是你儿女。”李瓶儿听了,终是爱缘不断。但题起来,辄流涕不止。

须臾过了五日,到廿七日早晨,雇了八名青衣白帽小童,大红销金棺与幡幢、雪盖、玉梅、雪柳围随,前首大红铭旌,题着“西门冢男之枢”。吴道官庙里,又差了十二众青衣小道童儿来,绕棺转咒《生神玉章》,动清乐送殡。众亲朋陪西门庆穿素服走至大街东口,将及门上,才上头口。西门庆恐怕李瓶儿到坟上悲痛,不叫他去。只是吴月娘、李娇儿、孟玉楼、潘金莲、大姐,家里五顶轿子,陪乔亲家母、大妗子和李桂儿、郑月儿、吴舜臣媳妇郑三姐往坟头去,留下孙雪娥、吴银儿并两个姑子在家与李瓶儿做伴儿。李瓶儿见不放他去,见棺材起身,送出到大门首,赶着棺材大放声,一口一声只叫:“不来家亏心的儿嚛!”叫的连声气破了。不防一头撞在门底下,把粉额磕伤,金钗坠地,慌的吴银儿与孙雪娥向前搊扶起来,劝归后边去了。到了房中,见炕上空落落的,只有他耍的那寿星博浪鼓儿还挂在床头上,想将起来,拍了桌子,又哭个不了。吴银儿在旁,拉着他手劝说道:“娘少哭了,哥哥已是抛闪你去了,那里再哭得活!你须自解自叹,休要只顾烦恼。”雪娥道:“你又年少青春,愁到明日养不出来也怎的?这里墙有缝,壁有眼,俺每不好说的。他使心用心,反累已身。他将你孩子害了,教他一还一报,问他要命。不知你我被他活埋了几遭了!只要汉子常守着他便好,到人屋里睡一夜儿,他就气生气死。早是前者,你每都知道,汉子等闲不到我后边,才到了一遭儿,你看他就背地里唧喳成一块,对着他姐儿每说我长道我短。俺每也不言语,每日洗眼儿看着他。这个淫妇,到明日还不知怎么死哩!”李瓶儿道:“罢了,我也惹了一身病在这里,不知在今日明日死,和他也争执不得了,随他罢!”

正说着,只见奶子如意儿向前跪下,哭道:“小媳妇有句活,不敢对娘说——今日哥儿死了,乃是小媳妇没造化。只怕往后爹与大娘打发小媳妇出去,小媳妇男子汉又没了,那里投奔?”李瓶儿见他这般说,又心中伤痛起来,便道:“怪老婆,孩子便没了,我还没死哩!总然我到明日死了,你恁在我手下一场,我也不教你出门。往后你大娘生下哥儿小姐来,交你接了奶,就是一般了。你慌乱的是甚么?”那如意儿方才不言语了。李瓶儿良久又悲恸哭起来,雪娥与吴银儿两个又解劝说道:“你肚中吃了些甚么,只顾哭了去!”一面叫绣春后边拿了饭来,摆在桌上,陪他吃。那李瓶儿怎生咽下去!只吃了半瓯儿,就丢下不吃了。

西门庆在坟上,叫徐先生画了穴,把官哥儿就埋在先头陈氏娘怀中,抱孙葬了。那日乔大户井众亲戚都有祭祀,就在新盖卷棚管待饮酒一日。来家,李瓶儿与月娘、乔大户娘子、大妗子磕着头又哭了。向乔大户娘子说道:“亲家,谁似奴养的孩儿不气长,短命死了。既死了,累你家姐姐做了望门寡,劳而无功,亲家休要笑话。”乔大户娘子说道:“亲家怎的这般说话?孩儿每各人寿数,谁人保的后来的事!常言:先亲后不改。亲家每又不老,往后愁没子孙?须要慢慢来。亲家也少要烦恼了。”说毕,作辞回家去了。

西门庆在前厅教徐先生洒扫,各门上都贴辟非黄符。死者煞高三丈,向东北方而去,遇日游神冲回不出,斩之则吉,亲人不忌。西门庆拿出一匹大布、二两银子谢了徐先生,管待出门。晚夕入李瓶儿房中陪他睡。夜间百般言语温存。见官哥儿的戏耍物件都还在跟前,恐怕这瓶儿看见思想烦恼,都令迎春拿到后边去了。正是:

\[
思想娇儿昼夜啼,寸心如割命悬丝。
世间万般哀苦事,除非死别共生离。
\]
