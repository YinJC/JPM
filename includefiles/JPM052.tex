%# -*- coding:utf-8 -*-
%%%%%%%%%%%%%%%%%%%%%%%%%%%%%%%%%%%%%%%%%%%%%%%%%%%%%%%%%%%%%%%%%%%%%%%%%%%%%%%%%%%%%


\chapter{应伯爵山洞戏春娇\KG 潘金莲花园调爱婿}


诗曰:

\[
春楼晓日珠帘映,红粉春妆宝镜催。
已厌交欢怜旧枕,相将游戏绕池台。
坐时衣带萦纤草,行处裙裾扫落梅。
更道明朝不当作,相期共斗管弦来。
\]

话说那日西门庆在夏提刑家吃酒,见宋巡按送礼,他心中十分欢喜。夏提刑亦敬重不同往日,拦门劝酒,吃至三更天气才放回家。潘金莲又早向灯下除去冠儿,设放衾枕,薰香澡牝等候。西门庆进门,接着,见他酒带半酣,连忙替他脱衣裳。春梅点茶吃了,打发上床歇息。见妇人脱得光赤条身子,坐在床沿,低垂着头,将那白生生腿儿横抱膝上缠脚,换了双大红平底睡鞋儿。西门庆一见,淫心辄起,麈柄挺然而兴。因问妇人要淫器包儿,妇人忙向褥子底下摸出来递与他。西门庆把两个托子都带上,一手搂过妇人在怀里,因说:“你达今日要和你干个‘后庭花儿’,你肯不肯?”那妇人瞅了一眼,说道:“好个没廉耻冤家,你成日和书童儿小厮干的不值了,又缠起我来了,你和那奴才干去不是!”西门庆笑道:“怪小油嘴,罢么!你若依了我,又稀罕小厮做甚么?你不知你达心里好的是这桩儿,管情放到里头去就过了。”妇人被他再三缠不不过,说道:“奴只怕挨不得你这大行货。你把头子上圈去了,我和你耍一遭试试。”西门庆真个除去硫磺圈,根下只束着银托子,令妇人马爬在床上,屁股高蹶,将唾津涂抹在龟头上,往来濡研顶入。龟头昂健,半晌仅没其棱。妇人在下蹙眉隐忍,口中咬汗巾子难捱,叫道:“达达慢着些。这个比不的前头,撑得里头热炙火燎的疼起来。”这西门庆叫道:“好心肝,你叫着达达,不妨事。到明日买一套好颜色妆花纱衣服与你穿。”妇人道:“那衣服倒也有在,我昨日见李桂姐穿的那玉色线掐羊皮挑的金油鹅黄银条纱裙子,倒好看,说是里边买的。他每都有,只我没这裙子。倒不知多少银子,你倒买一条我穿罢了。”西门庆道:“不打紧,我到明日替你买。”一壁说着,在上颇作抽拽,只顾没棱露脑,浅抽深送不已。妇人回首流眸叫道:“好达达,这里紧着人疼的要不的,如何只顾这般动作起来了?我央及你,好歹快些丢了罢!”这西门庆不听,且扶其股,玩其出入之势。一面口中呼道:“潘五儿,小淫妇儿,你好生浪浪的叫着达达,哄出你达达\textuni{379E}儿出来罢。”那妇人真个在下星眼朦胧,莺声款掉,柳腰款摆,香肌半就,口中艳声柔语,百般难述。良久,西门庆觉精来,两手扳其股,极力而\textuni{22D5E}之,扣股之声响之不绝。那妇人在下边呻吟成一块,不能禁止。临过之时,西门庆把妇人屁股只一扳,麈柄尽没至根,直抵于深异处,其美不可当。于是怡然感之,一泄如注。妇人承受其精,二体偎贴。良久拽出麈柄,但见猩红染茎,蛙口流涎,妇人以帕抹之,方才就寝。一宿晚景题过。

次日,西门庆早晨到衙门中回来,有安主事、黄主事那里差人来下请书,二十二日在砖厂刘太监庄上设席,请早去。西门庆打发来人去了,从上房吃了粥,正出厅来,只见篦头的小周儿扒倒地下磕头。西门庆道:“你来的正好,我正要篦篦头哩。”于是走到翡翠轩小卷棚内,坐在一张凉椅儿上,除了巾帻,打开头发。小周儿铺下梳篦家活,与他篦头栉发。观其泥垢,辨其风雪,跪下讨赏钱,说:“老爹今岁必有大迁转,发上气色甚旺。”西门庆大喜。篦了头,又叫他取耳,掐捏身上。他有滚身上一弄儿家活,到处与西门庆滚捏过,又行导引之法,把西门庆弄的浑身通泰。赏了他五钱银子,教他吃了饭,伺候着哥儿剃头。西门庆就在书房内,倒在大理石床上就睡着了。

那日杨姑娘起身,王姑子与薛姑子要家去。吴月娘将他原来的盒子都装了些蒸酥茶食,打发起身。两个姑子,每人都是五钱银子,两个小姑子,与了他两匹小布儿,管待出门。薛姑子又嘱咐月娘:“到了壬子日把那药吃了,管情就有喜事。”月娘道:“薛爷,你这一去,八月里到我生日,好来走走,我这里盼你哩。”薛姑子合掌问讯道:“打搅。菩萨这里,我到那日一定来。”于是作辞。月娘众人都送到大门首。月娘与大妗子回后边去了。只有玉楼、金莲、瓶儿、西门大姐、李桂姐抱着官哥儿,来到花园里游玩。李瓶儿道:“桂姐,你递过来,等我抱罢。”桂姐道:“六娘,不妨事,我心里要抱抱哥子。”玉楼道:“桂姐,你还没到你爹新收拾书房里瞧瞧哩。”到花园内,金莲见紫薇花开得烂熳,摘了两朵与桂姐戴。于是顺着松墙儿到翡翠轩,见里面摆设的床帐屏几、书画琴棋,极其潇洒。床上绡帐银钩,冰簟珊枕。西门庆倒在床上,睡思正浓。旁边流金小篆,焚着一缕龙涎。绿窗半掩,窗外芭蕉低映。潘金莲且在桌上掀弄他的香盒儿,玉楼和李瓶儿都坐在椅儿上,西门庆忽翻过身来,看刚见众妇人都在屋里,便道:“你每来做甚么?”金莲道:“桂姐要看看你的书房,俺每引他来瞧瞧。”那西门庆见他抱着官哥儿,又引逗了一回。忽见画童来说:“应二爹来了。”众妇人都乱走不迭,往李瓶儿那边去了。应伯爵走到松墙边,看见桂姐抱着官哥儿,便道:“好呀!李桂姐在这里。”故意问道:“你几时来?”那桂姐走了,说道:“罢么,怪花子!又不关你事,问怎的?”伯爵道:“好小淫妇儿,不关我事也罢,你且与我个嘴着。”于是搂过来就要亲嘴。被桂姐用手只一推,骂道:“贼不得人意怪攮刀子,若不是怕唬了哥子,我这一扇把子打的你……”西门庆走出来看见,说道:“怪狗才,看唬了孩儿!”因教书童:“你抱哥儿送与你六娘去。”那书童连忙接过来。奶子如意儿正在松墙拐角边等候,接的去了。伯爵和桂姐两个站着说话,问:“你的事怎样了?”桂姐道:“多亏爹这里可怜见,差保哥替我往东京说去了。”伯爵道:“好,好,也罢了。如此你放心些。”说毕,桂姐就往后边去了。伯爵道:“怪小淫妇儿,你过来,我还和你说话。”桂姐道:“我走走就来。”于是也往李瓶儿这边来了。

伯爵与西门庆才唱喏坐的。西门庆道:“昨日我在夏龙溪家吃酒,大巡宋道长那里差人送礼,送了一口鲜猪。我恐怕放不的,今早旋叫厨子来卸开,用椒料连猪头烧了。你休去,如今请谢子纯来,咱每打双陆,同享了罢。”一面使琴童儿:“快请你谢爹去。你说应二爹在这里。”琴童儿应诺去了。伯爵因问:“徐家银子讨来了不曾?”西门庆道:“贼没行止的狗骨秃,明日才先与二百五十两。你教他两个后日来,少的,我家里凑与他罢。”伯爵道:“这等又好了。怕不得他今日也买些鲜物儿来孝顺你。”西门庆道:“倒不消教他费心。”说了一回,西门庆问道:“老孙、祝麻子两个都起身去了不曾?”伯爵道:“自从李桂儿家拿出来,在县里监了一夜,第二日,三个一条铁索,都解上东京去了。到那里,没个清洁来家的!你只说成日图饮酒吃肉,好容易吃的果子儿!似这等苦儿,也是他受。路上这等大热天,着铁索扛着,又没盘缠,有甚么要紧。”西门庆笑道:“怪狗才,充军摆战的不过!谁教他成日跟着王家小厮只胡撞来!他寻的苦儿他受。”伯爵道:“哥说的有理。苍蝇不钻没缝的鸡蛋,他怎的不寻我和谢子纯?清的只是清,浑的只是浑。”

正说着,谢希大到了。唱毕喏坐下,只顾扇扇子。西门庆问道:“你怎的走恁一脸汗?”希大道:“哥别题起。今日平白惹了一肚子气。大清早晨,老孙妈妈子走到我那里,说我弄了他去。恁不合理的老淫妇!你家汉子成日摽着人在院里大酒大肉吃,大把挝了银子钱家去,你过阴去来?谁不知道!你讨保头钱,分与那个一分儿使也怎的?交我扛了两句走出来。不想哥这里呼唤。”伯爵道:“我刚才和哥不说,新酒放在两下里,清自清,浑自浑。当初咱每怎么说来?我说跟着王家小厮,到明日有一失。今日如何?撞到这网里,怨怅不的人!”西门庆道:“王家那小厮,有甚大气概?脑子还未变全,养老婆!还不勾俺每那咱撒下的,羞死鬼罢了!”伯爵道:“他曾见过甚么大头面目,比哥那咱的勾当,题起来把他唬杀罢了。”说毕,小厮拿茶上来吃了。西门庆道:“你两个打双陆。后边做着水面,等我叫小厮拿来咱每吃。”不一时,琴童来放桌儿。画童儿用方盒拿上四个小菜儿,又是三碟儿蒜汁、一大碗猪肉卤,一张银汤匙、三双牙箸。摆放停当,三人坐下,然后拿上三碗面来,各人自取浇卤,倾上蒜醋。那应伯爵与谢希大拿起箸来,只三扒两咽就是一碗。两人登时狠了七碗。西门庆两碗还吃不了,说道:“我的儿,你两个吃这些!”伯爵道:“哥,今日这面是那位姐儿下的?又好吃又爽口。”谢希大道:“本等卤打的停当,我只是刚才吃了饭了,不然我还禁一碗。”两个吃的热上来,把衣服脱了。见琴童儿收家活,便道:“大官儿,到后边取些水来,俺每漱漱口。”谢希大道:“温茶儿又好,热的烫的死蒜臭。”少顷,画童儿拿茶至。三人吃了茶,出来外边松墙外各花台边走了一道。只见黄四家送了四盒子礼来。平安儿掇进来与西门庆瞧:一盒鲜乌菱、一盒鲜荸荠、四尾冰湃的大鲥鱼、一盒枇杷果。伯爵看见说道:“好东西儿!他不知那里剜的送来,我且尝个儿着。”一手挝了好几个,递了两个与谢希大,说道:“还有活到老死,还不知此是甚么东西儿哩。”西门庆道:“怪狗才,还没供养佛,就先挝了吃?”伯爵道:“甚么没供佛,我且入口无赃着。”西门庆分咐:“交到后边收了。问你三娘讨三钱银子赏他。”伯爵问:“是李锦送来,是黄宁儿?”平安道:“是黄宁儿。”伯爵道:“今日造化了这狗骨秃了,又赏他三钱银子。”这里西门庆看着他两个打双陆不题。

且说月娘和桂姐、李娇儿、孟玉楼、潘金莲、李瓶儿、大姐,都在后边吃了饭,在穿廊下坐的。只见小周儿在影壁前探头舒脑的,李瓶儿道:“小周儿,你来的好。且进来与小大官儿剃剃头,他头发都长长了。”小周儿连忙向前都磕了头,说:“刚才老爹分咐,交小的进来与哥儿剃头。”月娘道:“六姐,你拿历头看看,好日子,歹日子,就与孩子剃头?”金莲便交小玉取了历头来,揭开看了一回,说道:“今日是四月廿一日,是个庚戌日,金定娄金狗当直,宜祭祀、官带、出行、裁衣、沐浴、剃头、修造、动土,宜用午时。——好日期。”月娘道:“既是好日子,叫丫头热水,你替孩儿洗头,教小周儿慢慢哄着他剃。”小玉在旁替他用汗巾儿接着头发,才剃得几刀,这官哥儿呱的怪哭起来。那小周连忙赶着他哭只顾剃,不想把孩子哭的那口气憋下去,不做声了,脸便胀的红了。李瓶儿唬慌手脚,连忙说:“不剃罢,不剃罢!”那小周儿唬的收不迭家活,往外没脚的跑。月娘道:“我说这孩予有些不长俊,护头。自家替他剪剪罢。平白教进来剃,剃的好么!”天假其便,那孩子憋了半日气,才放出声来。李瓶儿方才放心,只顾拍哄他,说道:“好小周儿,恁大胆!平白进来把哥哥头来剃了去了。剃的恁半落不合的,欺负我的哥哥。还不拿回来,等我打与哥哥出气。”于是抱到月娘跟前。月娘道:“不长俊的小花子儿,剃头耍了你了,这等哭?剩下这些,到明日做剪毛贼。”引逗了一回,李瓶儿交与奶子。月娘分咐:“且休与他奶吃,等他睡一回儿与他吃。”奶子抱的前边去了。只见来安儿进来取小周儿的家活,说唬的小周儿脸焦黄的。月娘问道:“他吃了饭不曾?”来安道:“他吃了饭。爹赏他五钱银子。”月娘教来安:“你拿一瓯子酒出去与他。唬着人家,好容易讨这几个钱!”小玉连忙筛了一盏,拿了一碟腊肉,教来安与他吃了去了。

吴月娘因教金莲:“你看看历头,几时是壬子日?”金莲看了,说道:“二十三日是壬子日,交芒种五月节。”便道:“姐姐你问他怎的?”月娘道:“我不怎的,问一声儿。”李桂姐接过历头来看了,说道:“这二十四日,苦恼是俺娘的生日!我不得在家。”月娘道:“前月初十日,是你姐姐生日,过了。这二十四日,可可儿又是你妈的生日了。原来你院中人家一日害两样病,做三个生日:日里害思钱病,黑夜思汉子的病。早晨是妈妈的生日,晌午是姐姐生日,晚夕是自家生日。——怎的都挤在一块儿?趁着姐夫有钱,撺掇着都生日了罢!”桂姐只是笑,不做声。只见西门庆使了画童儿来请,桂姐方向月娘房中妆点匀了脸,往花园中来。

卷棚内,又早放下八仙桌儿,桌上摆设两大盘烧猪肉并许多肴馔。众人吃了一回,桂姐在旁拿锺儿递酒,伯爵道:“你爹听着说,不是我索落你,人情儿已是停当了。你爹又替你县中说了,不寻你了。亏了谁?还亏了我再三央及你爹,他才肯了。平白他肯替你说人情去?随你心爱的甚么曲儿,你唱个儿我下酒,也是拿勤劳准折。”桂姐笑骂道:“怪硶花子,你虼蚤包网儿——好大面皮!爹他肯信你说话?”伯爵道:“你这贼小淫妇儿!你经还没念,就先打和尚。要吃饭,休恶了火头!你敢笑和尚投丈母,我就单丁摆布不起你这小淫妇儿?你休笑话,我半边俏还动的。”被桂姐把手中扇把子,尽力向他身上打了两下。西门庆笑骂道:“你这狗才,到明日论个男盗女娼,还亏了原问处。”笑了一回,桂姐慢慢才拿起琵琶,横担膝上,启朱唇,露皓齿,唱道:

\[
\cipaim{黄莺儿}谁想有这一种。减香肌,憔瘦损。镜鸾尘锁无心整。脂粉倦匀,花枝又懒簪。空教黛眉蹙破春山恨。
\]
伯爵道:“你两个当初好来,如今就为他耽些惊怕儿,也不该抱怨了。”桂姐道:“汗邪了你,怎的胡说!”——

\[
最难禁,谯楼上画角,吹彻了断肠声。
\]
伯爵道:“肠子倒没断,这一回来提你的断了线,你两个休提了。”被桂姐尽力打了一下,骂道:“贼攘刀的,今日汗邪了你,只鬼混人的。”——
\[
\cipaim{集资宾}幽窗静悄月又明,恨独倚帏屏。蓦听的孤鸿只在楼外鸣,把万愁又还题醒。更长漏永,早不觉灯昏香烬眠未成。他那里睡得安稳!
\]
伯爵道:“傻小淫妇儿,他怎的睡不安稳?又没拿了他去。落的在家里睡觉儿哩。你便在人家躲着,逐日怀着羊皮儿,直等东京人来,一块石头方落地。”桂姐被他说急了,便道:“爹,你看应花子,不知怎的,只发讪缠我。”伯爵道:“你这回才认的爹了?”桂姐不理他,弹着琵琶又唱:

\[
\cipaim{双声叠韵}思量起,思量起,怎不上心?无人处,无人处,泪珠儿暗倾。
\]
伯爵道:“一个人惯溺尿。一日,他娘死了,守孝打铺在灵前睡。晚了,不想又溺下了。人进来看见褥子湿,问怎的来,那人没的回答,只说:‘你不知,我夜间眼泪打肚里流出来了。’——就和你一般,为他声说不的,只好背地哭罢了。”桂姐道:“没羞的孩儿,你看见来?汗邪了你哩!”——

\[
我怨他,我怨他,说他不尽,谁知道这里先走滚。自恨我当初不合他认真。
\]
伯爵道:“傻小淫妇儿,如今年程,三岁小孩儿也哄不动,何况风月中子弟。你和他认真?你且住了,等我唱个南曲儿你听:‘风月事,我说与你听:如今年程,论不得假真。个个人古怪精灵,个个人久惯牢成,倒将计活埋把瞎缸暗顶。老虔婆只要图财,小淫妇儿少不得拽着脖子往前挣。苦似投河,愁如觅并。几时得把业罐子填完,就变驴变马也不干这营生。’”当下把桂姐说的哭起来了。被西门庆向伯爵头上打了一扇子,笑骂道:“你这搊断肠子的狗才!生生儿吃你把人就欧杀了。”因叫桂姐:“你唱,不要理他。”谢希大道:“应二哥,你好没趣!今日左来右去只欺负我这干女儿。你再言语,口上生个大疔疮。”那桂姐半日拿起琵琶,又唱:

\[
\cipaim{簇御林}人都道他志诚。
\]
伯爵才待言语,被希大把口按了,说道:“桂姐你唱,休理他!”桂姐又唱道:

\[
却原来厮勾引。眼睁睁心口不相应。
\]
希大放了手,伯爵又说:“相应倒好了。心口里不相应,如今虎口里倒相应。不多,也只三两炷儿。”桂姐道:“白眉赤眼,你看见来?”伯爵道:“我没看见,在乐星堂儿里不是?”连西门庆众人都笑起来了。桂姐又唱:

\[
山盟海誓,说假道真,险些儿不为他错害了相思病。负人心,看伊家做作,如何教我有前程?
\]
伯爵道:“前程也不敢指望他,到明日,少不了他个招宣袭了罢。”桂姐又唱:

\[
\cipaim{琥珀猫儿坠}日疏日远,何日再相逢?枉了奴痴心宁耐等。想巫山云雨梦难成。薄情,猛拚今生和你凤拆鸾零。
\cipaim{尾声}冤家下得忒薄幸,割舍的将人孤另。那世里的恩情翻成做话柄。
\]

唱毕,谢希大道:“罢,罢。叫画童儿接过琵琶去,等我酬劳桂姐一杯酒儿,消消气罢。”伯爵道:“等我哺菜儿。我本领儿不济事,拿勤劳准折罢了。”桂姐道:“花子过去,谁理你!你大拳打了人,这回拿手来摸挲。”当下,希大一连递了桂姐三杯酒,拉伯爵道:“咱每还有那两盘双陆,打了罢。”于是二人又打双陆。西门庆递了个眼色与桂姐,就往外走。伯爵道:“哥,你往后边左,捎些香茶儿出来。头里吃了些蒜,这回子倒反恶泛泛起来了。”西门庆道:“我那里得香茶来!”伯爵道:“哥,你还哄我哩,杭州刘学官送了你好少儿,你独吃也不好。”西门庆笑的后边去了。桂姐也走出来,在太湖石畔推摘花儿戴,也不见了。伯爵与希大一连打了三盘双陆,等西门庆白不见出来。问画童儿:“你爹在后边做甚么哩?”画童儿道:“爹在后边,就出来了。”伯爵道:“就出来,有些古怪!”因交谢希大:“你这里坐着,等我寻他寻去。”那谢希大且和书童儿两个下象棋。

原来西门庆只走到李瓶儿房里,吃了药就出来了。在木香棚下看见李桂姐,就拉到藏春坞雪洞儿里,把门儿掩着,坐在矮床儿上,把桂姐搂在怀中,腿上坐的,一径露出那话来与他瞧,把桂姐唬了一跳。便问:“怎的就这般大?”西门庆悉把吃胡僧药告诉了一遍。先交他低垂粉颈,款启猩唇,品咂了一回。然后,轻轻搊起他两只小小金莲来,跨在两边胳膊上,抱到一张椅儿上,两个就干起来。不想应伯爵到各亭儿上寻了一遭,寻不着,打滴翠岩小洞儿里穿过去,到了木香棚,抹过葡萄架,到松竹深处,藏春坞边,隐隐听见有人笑声,又不知在何处。这伯爵慢慢蹑足潜踪,掀开帘儿,见两扇洞门儿虚掩,在外面只顾听觑。听见桂姐颤着声儿,将身子只顾迎播着西门庆,叫:“达达,快些了事罢,只怕有人来。”被伯爵猛然大叫一声,推开门进来,看见西门庆把桂姐扛着腿子正干得好。说道:“快取水来,泼泼两个搂心的,搂到一答里了!”李桂姐道:“怪攘刀子,猛的进来,唬了我一跳!”伯爵道:“快些儿了事?好容易!也得值那些数儿是的。怕有人来看见,我就来了。且过来,等我抽个头儿着。”西门庆便道:“怪狗才,快出去罢了,休鬼混!我只怕小厮来看见。”那应伯爵道:“小淫妇儿,你央及我央及儿。不然我就吆喝起来,连后边嫂子每都嚷的知道。你既认做干女儿了,好意教你躲住两日儿,你又偷汉子。教你了不成!”桂姐道:“去罢,应怪花子!”伯爵道:“我去罢?我且亲个嘴着。”于是按着桂姐亲了一个嘴,才走出来。西门庆道:“怪狗才,还不带上门哩。”伯爵一面走来把门带上,说道:“我儿,两个尽着捣,尽着捣,捣吊底也不关我事。”才走到那个松树儿底下,又回来说道:“你头里许我的香茶在那里?”西门庆道:“怪狗才,等住回我与你就是了,又来缠人!”那伯爵方才一直笑的去了。桂姐道:“好个不得人意的攮刀子!”这西门庆和那桂姐两个,在雪洞内足干勾一个时辰,吃了一枚红枣儿,才得了事,雨散云收。有诗为证:

\[
海棠技上莺梭急,绿竹阴中燕语频。
闲来付与丹青手,一段春娇画不成。
\]

少顷,二人整衣出来。桂姐向他袖子内掏出好些香茶来袖了。西门庆使的满身香汗,气喘吁吁,走来马缨花下溺尿。李桂姐腰里摸出镜子来,在月窗上搁着,整云理鬓,往后边去了。

西门庆走到李瓶儿房里,洗洗手出来。伯爵问他要香茶,西门庆道:“怪花子,你害了痞,如何只鬼混人!”每人掐了一撮与他。伯爵道:“只与我这两个儿!由他,由他!等我问李家小淫妇儿要。”正说着,只见李铭走来磕头。伯爵道:“李日新在那里来?你没曾打听得他每的事怎么样儿了?”李铭道:“俺桂姐亏了爹这里。这两日,县里也没人来催,只等京中示下哩。”伯爵道:“齐家那小老婆子出来了?”李铭道:“齐香儿还在王皇亲宅内躲着哩。桂姐在爹这里好,谁人敢来寻?”伯爵道:“要不然也费手,亏我和你谢爹再三央劝你爹:‘你不替他处处儿,教他那里寻头脑去!’”李铭道:“爹这里不管,就了不成。俺三婶老人家,风风势势的,干出甚么事!”伯爵道:“我记的这几时是他生日,俺每会了你爹,与他做做生日。”李铭道:“爹每不消了。到明日事情毕了,三婶和桂姐,愁不请爹每坐坐?”伯爵道:“到其间,俺每补生日就是了。”因叫他近前:“你且替我吃了这锺酒着。我吃了这一日,吃不的了。”那李铭接过银把锺来,跪着一饮而尽。谢希大交琴童又斟了一锺与他。伯爵道:“你敢没吃饭?”桌上还剩了一盘点心,谢希大又拿两盘烧猪头肉和鸭子递与他。李铭双手接的,下边吃去了。伯爵用箸子又拨了半段鲥鱼与他,说道:“我见你今年还没食这个哩,且尝新着。”西门庆道:“怪狗才,都拿与他吃罢了,又留下做甚么?”伯爵道:“等住回吃的酒阑,上来饿了,我不会吃饭儿?你们那里晓得,江南此鱼一年只过一遭儿,吃到牙缝里剔出来都是香的。好容易!公道说,就是朝廷还没吃哩!不是哥这里,谁家有?”正说着,只见画童儿拿出四碟鲜物儿来:一碟乌菱、一碟荸荠、一碟雪藕、一碟枇杷。西门庆还没曾放到口里,被应伯爵连碟子都挝过去,倒的袖了。谢希大道:“你也留两个儿我吃。”也将手挝一碟子乌菱来。只落下藕在桌子上。西门庆掐了一块放在口内,别的与了李铭吃了。分付画童后边再取两个枇杷来赏李铭。李铭接的袖了,才上来拿筝弹唱。唱了一回,伯爵又出题目,叫他唱了一套《花药栏》。三个直吃到掌灯时候,还等后边拿出绿豆白米水饭来吃了,才起身。伯爵道:“哥,我晓得明日安主事请你,不得闲。李四、黄三那事,我后日会他来罢。”西门庆点头儿,二人也不等送,就去了。西门庆教书童看收家伙,就归后边孟玉楼房中歇去了。一宿无话。

到次日早起,也没往衙门中去,吃了粥,冠带骑马,书童、玳安两个跟随,出城南三十里,迳往刘太监庄上来赴席,不在话下。

潘金莲赶西门庆不在家,与李瓶儿计较,将陈敬济输的那三钱银子,又教李瓶儿添出七钱来,教来兴儿买了一只烧鸭、两只鸡、一钱银子下饭、一坛金华酒、一瓶白酒、一钱银子裹馅凉糕,教来兴儿媳妇整理端正。金莲对着月娘说:“大姐那日斗牌,赢了陈姐夫三钱银子,李大姐又添了些,今治了东道儿,请姐姐在花园里吃。”吴月娘就同孟玉楼、李娇儿、孙雪娥、大姐、桂姐众人,先在卷棚内吃了一回,然后拿酒菜儿,在山子上卧云亭下棋,投壶,吃酒耍子。月娘想起问道:“今日主人,怎倒不来坐坐?”大姐道:“爹又使他往门外徐家催银子去了,也好待来也。”

不一时,陈敬济来到,向月娘众人作了揖,就拉过大姐一处坐下。向月娘说:“徐家银子讨了来了,共五封二百五十两,送到房里,玉箫收了。”于是传杯换盏,酒过数巡,各添春色。月娘与李娇儿、桂姐三个下棋,玉楼众人都起身向各处观花玩草耍子。惟金莲独自手摇着白团纱扇儿,往山子后芭蕉深处纳凉。因见墙角草地下一朵野紫花儿可爱,便走去要摘。不想敬济有心,一眼睃见,便悄悄跟来,在背后说道:“五娘,你老人家寻甚么?这草地上滑齑齑的,只怕跌了你,教儿子心疼。”那金莲扭回粉颈,斜睨秋波,带笑带骂道:“好个贼短命的油嘴,跌了我,可是你就心疼哩?谁要你管!你又跟了我来做甚么,也不怕人看着。”因问:“你买的汗巾儿怎了?”敬济笑嘻嘻向袖于中取出,递与他,说道:“六娘的都在这里了。”又道:“汗巾儿买了来,你把甚来谢我?”于是把脸子挨的他身边,被金莲举手只一推。不想李瓶儿抱着官哥儿,并奶子如意儿跟着,从松墙那边走来。见金莲手拿自团扇一动,不知是推敬济,只认做扑蝴蝶,忙叫道:“五妈妈,扑的蝴蝶儿,把官哥儿一个耍子。”慌的敬济赶眼不见,两三步就钻进山子里边去了。金莲恐怕李瓶儿瞧见,故意问道:“陈姐夫与了汗巾不曾?”李瓶儿道:“他还没有与我哩。”金莲道:“他刚才袖着,对着大姐姐不好与咱的,悄悄递与我了。”于是两个坐在芭蕉丛下花台石上,打开分了。两个坐了一回,李瓶儿说道:“这答儿里到且是荫凉。”因使如意儿:“你去叫迎春屋里取孩子的小枕头并凉席儿来,就带了骨牌来,我和五娘在这里抹回骨牌儿。你就在屋里看罢。”如意儿去了。

不一时,迎春取了枕席并骨牌来。李瓶儿铺下席,把官哥儿放在小枕头儿上躺着,教他顽耍,他便和金莲抹牌。抹了一回,交迎春往屋里拿一壶好茶来。不想盂玉楼在卧云亭上看见,点手儿叫李瓶儿说:“大姐姐叫你说句话儿。”李瓶儿撇下孩子,教金莲看着:“我就来。”那金莲记挂敬济在洞儿里,那里又去顾那孩子,赶空儿两三步走入洞门首,教敬济,说:“没人,你出来罢。”敬济便叫妇人进去瞧蘑菇:“里面长出这些大头蘑菇来了。”哄的妇人入到洞里,就折叠腿跪着,要和妇人云雨。两个正接着亲嘴。也是天假其便,李瓶儿走到亭子上,月娘说:“孟三姐和桂姐投壶输了,你来替他投两壶儿。”李瓶儿道:“底下没人看孩子哩。”玉楼道:“左右有六姐在那里,怕怎的。”月娘道:“孟三姐,你去替他看看罢。”李瓶儿道:“三娘累你,亦发抱了他来罢。”教小玉:“你去就抱他的席和小枕头儿来。”那小玉和玉楼走到芭蕉丛下,孩子便躺在席上,蹬手蹬脚的怪哭,并不知金莲在那里。只见旁边一个大黑猫,见人来,一溜烟跑了。玉楼道:“他五娘那里去了?耶嚛,耶嚛!把孩子丢在这里,吃猫唬了他了。”那金莲连忙从雪洞儿里钻出来,说道:“我在这里净了净手,谁往那里去来!那里有猫唬了他?白眉赤眼的!”那玉楼也更不往洞里看,只顾抱了官哥儿,拍哄着他往卧云亭儿上去了。小玉拿着枕席跟的去了。金莲恐怕他学舌,随屁股也跟了来。月娘问:“孩子怎的哭?”玉楼道:“我去时,不知是那里一个大黑猫蹲在孩子头跟前。”月娘说:“干净唬着孩儿。”李瓶儿道,“他五娘看着他哩。”玉楼道:“六姐往洞儿里净手去来。”金莲走上来说:“三姐,你怎的恁白眉赤眼儿的?那里讨个猫来!他想必饿了,要奶吃哭,就赖起人来。”李瓶几见迎春拿上茶来,就使他叫奶子来喂哥儿奶。

陈敬济见无人,从洞儿钻出来,顺着松墙儿转过卷棚,一直往外去了。正是:

\[
两手劈开生死路。一身跳出是非门。
\]


月娘见孩子不吃奶,只是哭,分咐李瓶儿:“你抱他到屋里,好好打发他睡罢。”于是也不吃酒,众人都散了。原来陈敬济也不曾与潘金莲得手,事情不巧,归到前边厢房中,有些咄咄不乐。正是:

\[
无可奈何花落去,似曾相识燕归来。
\]

