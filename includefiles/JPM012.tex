%# -*- coding:utf-8 -*-
%%%%%%%%%%%%%%%%%%%%%%%%%%%%%%%%%%%%%%%%%%%%%%%%%%%%%%%%%%%%%%%%%%%%%%%%%%%%%%%%%%%%%


\chapter{潘金莲私仆受辱\KG 刘理星魇胜求财}


诗曰:

\[
可怜独立树,枝轻根亦摇。虽为露所浥,复为风所飘。
锦衾襞不开,端坐夜及朝。是妾愁成瘦,非君重细腰。
\]

话说西门庆在院中贪恋桂姐姿色,约半月不曾来家。吴月娘使小厮拿马接了数次,李家把西门庆衣帽都藏过,不放他起身。丢的家中这些妇人都闲静了。别人犹可,惟有潘金莲这妇人,青春未及三十岁,欲火难禁一丈高。每日打扮的粉妆玉琢,皓齿朱唇,无日不在大门首倚门而望,只等到黄昏。到晚来归入房中,粲枕孤帏,凤台无伴,睡不着,走来花园中,款步花苔。看见那月洋水底,便疑西门庆情性难拿;偶遇着玳瑁猫儿交欢,越引逗的他芳心迷乱。当时玉楼带来一个小厮,名唤琴童,年约十六岁,才留起头发,生的眉目清秀,乖滑伶俐。西门庆教他看管花园,晚夕就在花园门首一间小耳房内安歇。金莲和玉楼白日里常在花园亭子上一处做针指或下棋。这小厮专一献小殷勤,常观见西门庆来,就先来告报。以此妇人喜他,常叫他入房,赏酒与他吃。两个朝朝暮暮,眉来眼去,都有意了。

不想到了七月,西门庆生日将近。吴月娘见西门庆留恋烟花,因使玳安拿马去接。这潘金莲暗暗修了一柬帖,交付玳安,教:“悄悄递与你爹,说五娘请爹早些家去罢。”这玳安儿一直骑马到李家,只见应伯爵、谢希大、祝实念,孙寡嘴,常峙节众人,正在那里伴着西门庆,搂着粉头欢乐饮酒。西门庆看见玳安来到,便问:“你来怎麽?家中没事?”玳安道:“家中没事。”西门庆道:“前边各项银子,叫傅二叔讨讨,等我到家算帐。”玳安道:“这两日傅二叔讨了许多,等爹到家上帐。”西门庆道:“你桂姨那一套衣服,捎来不曾?”玳安道:“已捎在此。”便向毡包内取出一套红衫蓝裙,递与桂姐。桂姐道了万福,收了,连忙分付下边,管待玳安酒饭。那小厮吃了酒饭,复走来上边伺候。悄悄向西门庆耳边说道:“五娘使我捎了个帖儿在此。请爹早些家去。”西门庆才待用手去接,早被李桂姐看见,只道是西门庆那个表子寄来的情书,一手挝过来,拆开观看,却是一幅回文锦笺,上写着几行墨迹。桂姐递与祝实念,教念与他听。这祝实念见上面写词一首,名《落梅风》,念道:

\[
黄昏想,白日思,盼杀人多情不至。因他为他憔悴死,可怜也绣衾独自!灯将残,人睡也,空留得半窗明月。眠心硬,浑似铁,这凄凉怎捱今夜?
\]
下书:“爱妾潘六儿拜。”那桂姐听毕,撇了酒席,走入房中,倒在床上,面朝里边睡了。西门庆见桂姐恼了,把帖子扯的稀烂,众人前把玳安踢了两脚。请桂姐两遍不来,慌的西门庆亲自进房,抱出他来,说道:“分付带马回去,家中那个淫妇使你来,我这一到家,都打个臭死!”玳安只得含泪回家。西门庆道:“桂姐,你休恼,这帖子不是别人的,乃是我第五个小妾寄来,请我到家有些事儿计较,再无别故。”祝实念在旁戏道:“桂姐,你休听他哄你哩!这个潘六儿乃是那边院里新叙的一个表子,生的一表人物。你休放他去。”西门庆笑赶着打,说道:“你这贱天杀的,单管弄死了人,紧着他恁麻犯人,你又胡说。”李桂卿道:“姐夫差了,既然家中有人拘管,就不消梳笼人家粉头,自守着家里的便了。才相伴了多少时,便就要抛离了去。”应伯爵插口道:“说的有理。你两人都依我,大官人也不消家去,桂姐也不必恼。今日说过,那个再恁,每人罚二两银子,买酒咱大家吃。”于是西门庆把桂姐搂在怀中陪笑,一递一口儿饮酒。少倾,拿了七锺茶来,馨香可掬,每人面前一盏。应伯爵道:“我有个曲儿,单道这茶好处:

\[
\cipaim{朝天子}这细茶的嫩芽,生长在春风下。不揪不采叶儿楂,但煮着颜色大。绝品清奇,难描难画。口里儿常时呷,醉了时想他,醒来时爱他。原来一篓儿千金价。”
\]

谢希大笑道:“大官人使钱费物,不图这‘一搂儿’,却图些甚的?如今每人有词的唱词,不会词,每人说个笑话儿,与桂姐下酒。”就该谢希大先说,因说道:“有一个泥水匠,在院中墁地。老妈儿怠慢了他,他暗把阴沟内堵上块砖。落后天下雨,积的满院子都是水。老妈慌了,寻的他来,多与他酒饭,还秤了一钱银子,央他打水平。那泥水匠吃了酒饭,悄悄去阴沟内把那块砖拿出,那水登时出的罄尽。老妈便问作头:‘此是那里的病?’泥水匠回道:‘这病与你老人家的病一样,有钱便流,无钱不流。’”桂姐见把他家来伤了,便道:“我也有个笑话,回奉列位。有一孙真人,摆着筵席请人,却教座下老虎去请。那老虎把客人都路上一个个吃了。真人等至天晚,不见一客到。不一时老虎来,真人便问:‘你请的客人都那里去了?’老虎口吐人言:‘告师父得知,我从来不晓得请人,只会白嚼人。’”当下把众人都伤了。应伯爵道:“可见的俺们只是白嚼,你家孤老就还不起个东道?”于是向头上拨下一根闹银耳斡儿来,重一钱;谢希大一对镀金网巾圈,秤了秤重九分半;祝实念袖中掏出一方旧汗巾儿,算二百文长钱;孙寡嘴腰间解下一条白布裙,当两壶半酒;常峙节无以为敬,问西门庆借了一钱银子。都递与桂卿,置办东道,请西门庆和桂姐。那桂卿将银钱都付与保儿,买了一钱猪肉,又宰了一只鸡,自家又陪些小菜儿,安排停当。大盘小碗拿上来,众人坐下,说了一声动箸吃时,说时迟,那时快,但见:

\[
人人动嘴,个个低头。遮天映日,犹如蝗蚋一齐来;挤眼掇肩,好似饿牢才打出。这个抢风膀臂,如经年未见酒和肴;那个连三筷子,成岁不筵与席。一个汗流满面,却似与鸡骨秃有冤仇;一个油抹唇边,把猪毛皮连唾咽。吃片时,杯盘狼藉;啖顷刻,箸子纵横。这个称为食王元帅,那个号作净盘将军。酒壶番晒又重斟,盘馔已无还去探。正是:珍羞百味片时休,果然都送入五脏庙。
\]
当下众人吃得个净光王佛。西门庆与桂姐吃不上两锺酒,拣了些菜蔬,又被这伙人吃去了。那日把席上椅子坐折了两张,前边跟马的小厮,不得上来掉嘴吃,把门前供养的土地翻倒来,便剌了一泡屯谷都的热屎。临出门来,孙寡嘴把李家明间内供养的镀金铜佛,塞在裤腰里;应伯爵推斗桂姐亲嘴,把头上金琢针儿戏了;谢希大把西门庆川扇儿藏了;祝实念走到桂卿房里照面,溜了他一面水银镜子。常峙节借的西门庆一钱银子,竞是写在嫖账上了。原来这起人,只伴着西门庆玩耍,好不快活。有诗为证:

\[
工妍掩袖媚如猱,乘兴闲来可暂留。
若要死贪无厌足,家中金钥教谁收?
\]

按下众人簇拥着西门庆饮酒不题。单表玳安回马到家,吴月娘和孟玉楼、潘金莲正在房坐的,见了便问玳安:“你去接爹来了不曾?”玳安哭的两眼红红的,说道:被爹踢骂了小的来了。爹说那个再使人接,来家都要骂。”月娘便道:“你看恁不合理,不来便了,如何又骂小厮?”孟玉楼道:“你踢将小厮便罢了,如何连俺们都骂将来?”潘金莲道:“十个九个院中淫妇,和你有甚情实!常言说的好:船载的金银,填不满烟花寨。”金莲只知说出来,不防李娇儿见玳安自院中来家,便走来窗下潜听。见金莲骂他家千淫妇万淫妇,暗暗怀恨在心。从此二人结仇,不在话下。正是:

\[
甜言美语三冬暖,恶语伤人六月寒。
\]

不说李娇儿与潘金莲结仇。单表金莲归到房中,捱一刻似三秋,盼一时如半夏。知道西门庆不来家,把两个丫头打发睡了,推往花园中游玩,将琴童叫进房与他酒吃。把小厮灌醉了,掩上房门,褪衣解带,两个就干做一处。但见:

\[
一个不顾纲常贵贱,一个那分上下高低。一个色胆歪邪,管甚丈夫利害;一个淫心荡漾,纵他律法明条。百花园内,翻为快活排场;主母房中,变作行乐世界。霎时一滴驴精髓,倾在金莲玉体中。
\]
自此为始,每夜妇人便叫琴童进房如此。未到天明,就打发出来。背地把金裹头簪子两三根带在头上,又把裙边带的锦香囊葫芦儿也与了他。岂知这小厮不守本分,常常和同行小厮街上吃酒耍钱,颇露机关。常言:若要不知,除非莫为。有一日,风声吹到孙雪娥、李娇儿耳朵内,说道:“贼淫妇,往常假撇清,如何今日也做出来了?”齐来告月娘。月娘再三不信,说道:“不争你们和他合气,惹的孟三姐不怪?只说你们挤撮他的小厮。”说的二人无言而退。落后妇人夜间和小厮在房中行事,忘记关厨房门,不想被丫头秋菊出来净手,看见了。次日传与后边小玉,小玉对雪娥说。雪娥同李娇儿又来告诉月娘如此这般:“他屋里丫头亲口说出来,又不是俺们葬送他。大娘不说,俺们对他爹说。若是饶了这个淫妇,非除饶了蝎子!”

此时正值七月二十七日,西门庆从院中来家上寿。月娘道:“他才来家,又是他好日子,你们不依我,只顾说去!等他反乱将起来,我不管你。”二人不听月娘,约的西门庆进入房中,齐来告诉金莲在家怎的养小厮一节。这西门庆不听万事皆休,听了怒从心上起,恶向胆边生。走到前边坐下,一片声叫琴童儿。早有人报与潘金莲。金莲慌了手脚,使春梅忙叫小厮到房中,嘱咐千万不要说出来,把头上簪子都拿过来收了。着了慌,就忘解了香囊葫芦下来。被西门庆叫到前厅跪下,分付三四个小厮,选大板子伺候。西门庆道:“贼奴才,你知罪么?”那琴童半日不敢言语。西门庆令左右:“拨下他簪子来,我瞧!”见没了簪子,因问:“你戴的金裹头银簪子,往那里去了?”琴童道:“小的并没甚银簪子。”西门庆道:“奴才还捣鬼!与我旋剥了衣服,拿板子打!”当下两三个小厮扶侍一个,剥去他衣服,扯了裤子。见他身底下穿着玉色绢裈儿,裈儿带上露出锦香囊葫芦儿。西门庆一眼看见,便叫:“拿上来我瞧!”认的是潘金莲裙边带的物件,不觉心中大怒,就问他:“此物从那里得来?你实说是谁与你的?”唬的小厮半日开口不得,说道:“这是小的某日打扫花园,在花园内拾的。并不曾有人与我。”西门庆越怒,切齿喝令:“与我捆起来着实打!”当下把琴童绷子绷着,打了三十大棍,打得皮开肉绽,鲜血顺腿淋漓。又叫来保:“把奴才两个鬓毛与我撏了!赶将出去,再不许进门!”那琴童磕了头,哭哭啼啼出门去了。

潘金莲在房中听见,如提冷水盆内一般。不一时,西门庆进房来,吓的战战兢兢,浑身无了脉息,小心在旁扶侍接衣服,被西门庆兜脸一个耳刮子,把妇人打了一交。分付春梅:“把前后角门顶了,不放一个人进来!”拿张小椅儿,坐在院内花架儿底下,取了一根马鞭子,拿在手里,喝令:“淫妇,脱了衣裳跪着!”那妇人自知理亏,不敢不跪,真个脱去了上下衣服,跪在面前,低垂粉面,不敢出一声儿。西门庆便问:“贼淫妇,你休推梦里睡里,奴才我已审问明白,他一一都供出来了。你实说,我不在家,你与他偷了几遭?”妇人便哭道:“天那,天那!可不冤屈杀了我罢了!自从你不在家半个来月,奴白日里只和孟三儿一处做针指,到晚夕早关了房门就睡了。没勾当,不敢出这角门边儿来。你不信,只问春梅便了。有甚和盐和醋,他有个不知道的?”因叫春梅:“姐姐你过来,亲对你爹说。”西门庆骂道:“贼淫妇!有人说你把头上金裹头簪子两三根都偷与了小厮,你如何不认?”妇人道:“就屈杀了奴罢了!是那个不逢好死的嚼舌根的淫妇,嚼他那旺跳身子。见你常时进奴这屋里来歇,无非都气不愤,拿这有天没日头的事压枉奴。就是你与的簪子,都有数儿,一五一十都在,你查不是!我平白想起甚么来与那奴才?好成材的奴才,也不枉说的,恁一个尿不出来的毛奴才,平空把我篡一篇舌头!”西门庆道:“簪子有没罢了。”因向袖中取出那香囊来,说道:“这个是你的物件儿,如何打小厮身底下捏出来?你还口强甚么?”说着纷纷的恼了,向他白馥馥香肌上,飕的一马鞭子来,打的妇人疼痛难忍,眼噙粉泪,没口子叫道:“好爹爹,你饶了奴罢!你容奴说便说,不容奴说,你就打死了奴,也只臭烂了这块地。这个香囊葫芦儿,你不在家,奴那日同孟三姐在花园里做生活,因从木香棚下过,带儿系不牢,就抓落在地,我那里没寻,谁知这奴才拾了。奴并不曾与他。”只这一句,就合着琴童供称一样的话,又见妇人脱的光赤条条,花朵儿般身子,娇啼嫩语,跪在地下,那怒气早已钻入爪洼国去了,把心已回动了八九分,因叫过春梅,搂在怀中,问他:“淫妇果然与小厮有首尾没有?你说饶了淫妇,我就饶了罢。”那春梅撒娇撒痴,坐在西门庆怀里,说道:“这个,爹你好没的说!我和娘成日唇不离腮,娘肯与那奴才?这个都是人气不愤俺娘儿们,做作出这样事来。爹,你也要个主张,好把丑名儿顶在头上,传出外边去好听?”几句把西门庆说的一声儿没言语,丢了马鞭子,一面叫金莲起来,穿上衣服,分付秋菊看菜儿,放桌儿吃酒。这妇人满斟了一杯酒,双手递上去,跪在地下,等他锺儿。西门庆分付道:“我今日饶了你。我若但凡不在家,要你洗心改正,早关了门户,不许你胡思乱想。我若知道,并不饶你!”妇人道:“你分付,奴知道了。”又与西门庆磕了四个头,方才安坐儿,在旁陪坐饮酒。潘金莲平日被西门庆宠的狂了,今日讨这场羞辱在身上。正是:

\[
为人莫作妇人身,百年苦乐由他人。
\]

当下西门庆正在金莲房中饮酒,忽小厮打门,说:“前边有吴大舅、吴二舅、傅伙计、女儿、女婿,众亲戚送礼来祝寿。”方才撇了金莲,出前边陪待宾客。那时应伯爵、谢希大众人都有人情,院中李桂姐家亦使保儿送礼来。西门庆前边乱着收人家礼物,发柬请人,不在话下。

且说孟玉楼打听金莲受辱,约的西门庆不在房里,瞒着李娇儿、孙雪娥,走来看望。见金莲睡在床上,因问道:“六姐,你端的怎么缘故?告我说则个。”那金莲满眼流泪哭道:“三姐,你看小淫妇,今日在背地里白唆调汉子,打了我恁一顿。我到明日,和这两个淫妇冤仇结得有海深。”玉楼道:“你便与他有瑕玷,如何做作着把我的小厮弄出去了?六姐,你休烦恼,莫不汉子就不听俺们说句话儿?若明日他不进我房里来便罢,但到我房里来,等我慢慢劝他。”金莲道:“多谢姐姐费心。”一面叫春梅看茶来吃。坐着说了回话,玉楼告回房去了。至晚,西门庆因上房吴大妗子来了,走到玉楼房中宿歇。玉楼因说道:“你休枉了六姐心,六姐并无此事,都是日前和李娇儿、孙雪娥两个有言语,平白把我的小厮扎罚了。你不问个青红皂白,就把他屈了,却不难为他了!我就替他赌个大誓,若果有此事,大姐姐有个不先说的?”西门庆道:“我问春梅,他也是这般说。”玉楼道:“他今在房中不好哩,你不去看他看去?”西门庆道:“我知道,明日到他房中去。”当晚无话。

到第二日,西门庆正生日。有周守备、夏提刑、张团练、吴大舅许多官客饮酒,拿轿子接了李桂姐并两个唱的,唱了一日。李娇儿见他侄女儿来,引着拜见月娘众人,在上房里坐吃茶。请潘金莲见,连使丫头请了两遍,金莲不出来,只说心中不好。到晚夕,桂姐临家去,拜辞月娘。月娘与他一件云绢比甲儿、汗巾花翠之类,同李娇儿送出门首。桂姐又亲自到金莲花园角门首:“好歹见见五娘。”那金莲听见他来,使春梅把角门关得铁桶相似,说道:“娘分付,我不敢开。”这花娘遂羞讪满面而回,不题。

单表西门庆至晚进入金莲房内来,那金莲把云鬓不整,花容倦淡,迎接进房,替他脱衣解带,伺候茶汤脚水,百般殷勤扶侍。到夜里枕席欢娱,屈身忍辱,无所不至,说道:“我的哥哥,这一家谁是疼你的?都是露水夫妻,再醮货儿。惟有奴知道你的心,你知道奴的意。旁人见你这般疼奴,在奴身边的多,都气不愤,背地里驾舌头,在你跟前唆调。我的傻冤家!你想起甚么来,中人的拖刀之计,把你心爱的人儿这等下无情的折挫!常言道:家鸡打的团团转,野鸡打的贴天飞。你就把奴打死了,也只在这屋里。就是前日你在院里踢骂了小厮来,早是有大姐姐、孟三姐在跟前,我自不是说了一声,恐怕他家粉头掏渌坏了你身子,院中唱的一味爱钱,有甚情节?谁人疼你?谁知被有心的人听见,两个背地做成一帮儿算计我。自古人害人不死,天害人才害死了。往后久而自明,只要你与奴做个主儿便了。”几句把西门庆窝盘住了。是夜与他淫欲无度。

过了几日,西门庆备马,玳安、平安两个跟随,往院中来。却说李桂姐正打扮着陪人坐的,听见他来,连忙走进房去,洗了浓妆,除了簪环,倒在床上裹衾而卧。西门庆走到,坐了半日,老妈才出来,道了万福,让西门庆坐下,问道:“怎的姐夫连日不进来走走?”西门庆道:“正是因贱日穷冗,家中无人。”虔婆道:“姐儿那日打搅。”西门庆道:“怎的那日桂卿不来走走?”虔婆道:“桂卿不在家,被客人接去店里。这几日还不放了来。”说了半日话,才拿茶来陪着吃了。西门庆便问:“怎的不见桂姐?”虔婆道:“姐夫还不知哩,小孩儿家,不知怎的,那日着了恼,来家就不好起来,睡倒了。房门儿也不出,直到如今。姐夫好狠心,也不来看看姐儿。”西门庆道:“真个?我通不知。”因问:“在那边房里?我看看去。”虔婆道:“在他后边卧房里睡。”慌忙令丫鬟掀帘子。西门庆走到他房中,只见粉头乌云散乱,粉面慵妆,裹被坐在床上,面朝里,见了西门庆,不动一动儿。西门庆道:“你那日来家,怎的不好?”也不答应。又问:“你着了谁人恼,你告我说。”问了半日,那桂姐方开言说道:“左右是你家五娘子。你家中既有恁好的迎欢卖俏,又来稀罕俺们这样淫妇做甚么?俺们虽是门户中出身,跷起脚儿,比外边良人家不成的货色儿高好些!我前日又不是供唱,我也送人情去。大娘到见我甚是亲热,又与我许多花翠衣服。待要不请他见,又说俺院中没礼法。闻说你家有五娘子,当即请他拜见,又不出来。家来同俺姑娘又辞他去,他使丫头把房门关了。端的好不识人敬重!”西门庆道:“你到休怪他。他那日本等心中不自在,他若好时,有个不出来见你的?这个淫妇,我几次因他咬群儿,口嘴伤人,也要打他哩!”桂姐反手向西门庆脸上一扫,说道:“没羞的哥儿,你就打他?”西门庆道:“你还不知我手段,除了俺家房下,家中这几个老婆丫头,但打起来也不善,着紧二三十马鞭子还打不下来。好不好还把头发都剪了。”桂姐道:“我见砍头的,没见吹嘴的,你打三个官儿,唱两个喏,谁见来?你若有本事,到家里只剪下一柳子头发,拿来我瞧,我方信你是本司三院有名的子弟。”西门庆道:“你敢与我排手?”那桂姐道:“我和你排一百个手。”当日西门庆在院中歇了一夜,到次日黄昏时分,辞了桂姐,上马回家。桂姐道:“哥儿,你这一去,没有这物件儿,看你拿甚嘴脸见我!”

这西门庆吃他激怒了几句话,归家已是酒酣,不往别房里去,迳到潘金莲房内来。妇人见他有酒了,加意用心伏侍。问他酒饭都不吃。分付春梅把床上枕席拭抹干净,带上门出去。他便坐在床上,令妇人脱靴。那妇人不敢不脱。须臾,脱了靴,打发他上床。西门庆且不睡,坐在一只枕头上,令妇人褪了衣服,地下跪着。那妇人吓的捏两把汗,又不知因为甚么,于是跪在地下,柔声痛哭道:“我的爹爹!你透与奴个伶俐说话,奴死也甘心。饶奴终日恁提心吊胆,陪着一千个小心,还投不着你的机会,只拿钝刀子锯处我,教奴怎生吃受?”西门庆骂道:“贱淫妇,你真个不脱衣裳,我就没好意了!”因叫春梅:“门背后有马鞭子,与我取了来!”那春梅只顾不进房来,叫了半日,才慢条厮礼推开房门进来。看见妇人跪在床地平上,向灯前倒着桌儿下,由西门庆使他,只不动身。妇人叫道:“春梅,我的姐姐,你救我救儿,他如今要打我。”西门庆道:“小油嘴儿,你不要管他。你只递马鞭子与我打这淫妇。”春梅道:“爹,你怎的恁没羞!娘干坏了你甚么事儿?你信淫妇言语,平地里起风波,要便搜寻娘?还教人和你一心一计哩!你教人有那眼儿看得上你!倒是我不依你。”拽上房门,走在前边去了。那西门庆无法可处,倒呵呵笑了,向金莲道:“我且不打你。你上来,我问你要椿物儿,你与我不与我?”妇人道:“好亲亲,奴一身骨朵肉儿都属了你,随要甚么,奴无有不依随的。不知你心里要甚么儿?”西门庆道:“我要你顶上一柳儿好头发。”妇人道:“好心肝!奴身上随你怎的拣着烧遍了也依,这个剪头发却依不的,可不吓死了我罢了。奴出娘胞儿,活了二十六岁,从没干这营生。打紧我顶上这头发近来又脱了好些,只当可怜见我罢。”西门庆道:“你只怪我恼,我说的你就不依。”妇人道:“我不依你,再依谁?”因问:“你实对奴说,要奴这头发做甚么?”西门庆道:“我要做网巾。”妇人道:“你要做网巾,奴就与你做,休要拿与淫妇,教他好压镇我。”西门庆道:“我不与人便了,要你发儿做顶线儿。”妇人道:“你既要做顶线,待奴剪与你。”当下妇人分开头发,西门庆拿剪刀,按妇人顶上,齐臻臻剪下一大柳来,用纸包放在顺袋内。妇人便倒在西门庆怀中,娇声哭道:“奴凡事依你,只愿你休忘了心肠,随你前边和人好,只休抛闪了奴家!”是夜与他欢会异常。

到次日,西门庆起身,妇人打发他吃了饭,出门骑马,迳到院里。桂姐便问:“你剪的他头发在那里?”西门庆道:“有,在此。”便向茄袋内取出,递与桂姐。打开看,果然黑油也一般好头发,就收在袖中。西门庆道:“你看了还与我,他昨日为剪这头发,好不烦难,吃我变了脸恼了,他才容我剪下这一柳子来。我哄他,只说要做网巾顶线儿,迳拿进来与你瞧。可见我不失信。”桂姐道:“甚么稀罕货,慌的恁个腔儿!等你家去,我还与你。比是你恁怕他,就不消剪他的来了。”西门庆笑道:“那里是怕他!恁说我言语不的了。”桂姐一面叫桂卿陪着他吃酒,走到背地里,把妇人头发早絮在鞋底下,每日踹踏,不在话下。却把西门庆缠住,连过了数日,不放来家。

金莲自从头发剪下之后,觉道心中不快,每日房门不出,茶饭慵餐。吴月娘使小厮请了家中常走看的刘婆子来看视,说:“娘子着了些暗气,恼在心中,不能回转,头疼恶心,饮食不进。”一面打开药包来,留了两服黑丸子药儿:“晚上用姜汤吃。”又说:“我明日叫我老公来,替你老人家看看今岁流年,有灾没灾。”金莲道:“原来你家老公也会算命?”刘婆道:“他虽是个瞽目人,到会两三椿本事:第一善阴阳算命,与人家禳保;第二会针灸收疮;第三椿儿不可说,——单管与人家回背。”妇人问道:“怎么是回背?”刘婆子道:“比如有父子不和,兄弟不睦,大妻小妻争斗,教了俺老公去说了,替他用镇物安镇,画些符水与他吃了,不消三日,教他父子亲热,兄弟和睦,妻妾不争。若人家买卖不顺溜,田宅不兴旺者,常与人开财门发利市。治病洒扫,禳星告斗都会。因此人都叫他做刘理星。也是一家子,新娶个媳妇儿是小人家女儿,有些手脚儿不稳,常偷盗婆婆家东西往娘家去。丈夫知道,常被责打。俺老公与他回背,画了一道符,烧灰放在水缸下埋着,合家大小吃了缸内水,眼看媳妇偷盗,只象没看见一般。又放一件镇物在枕头内,男子汉睡了那枕头,好似手封住了的,再不打他了。”那金莲听见遂留心,便呼丫头,打发茶汤点心与刘婆吃。临去,包了三钱药钱,另外又秤了五钱,要买纸扎信信物。明日早饭时叫刘瞎来烧神纸。那婆子作辞回家。

到次日,果然大清早晨,领贼瞎迳进大门往里走。那日西门庆还在院中,看门小厮便问:“瞎子往那里走?”刘婆道:“今日与里边五娘烧纸。”小厮道:“既是与五娘烧纸,老刘你领进去。仔细看狗。”这婆子领定,迳到潘金莲卧房明间内,等了半日,妇人才出来。瞎子见了礼,坐下。妇人说与他八字,贼瞎用手捏了捏,说道:“娘子庚辰年,庚寅月,乙亥日,己丑时。初八日立春,已交正月算命。依子平正论,娘子这八字,虽故清奇,一生不得夫星济,子上有些防碍。乙木生在正月间,亦作身旺论,不克当自焚。又两重庚金,羊刃大重,夫星难为,克过两个才好。”妇人道:“已克过了。”贼瞎子道:“娘子这命中,休怪小人说,子平虽取煞印格,只吃了亥中有癸水,丑中又有癸水,水太多了,冲动了只一重巳土,官煞混杂。论来,男人煞重掌威权,女子煞重必刑夫。所以主为人聪明机变,得人之宠。只有一件,今岁流年甲辰,岁运并临,灾殃立至。命中又犯小耗勾绞,两位星辰打搅,虽不能伤,却主有比肩不和,小人嘴舌,常沾些啾唧不宁之状。”妇人听了,说道:“累先生仔细用心,与我回背回背。我这里一两银子相谢先生,买一盏茶吃。奴不求别的,只愿得小人离退,夫主爱敬便了。”一面转入房中,拔了两件首饰递与贼瞎。贼瞎收入袖中,说道:“既要小人回背,用柳木一块,刻两个男女人形,书着娘子与夫主生辰八字,用七七四十九根红线扎在一处。上用红纱一片,蒙在男子眼中,用艾塞其心,用针钉其手,下用胶粘其足,暗暗埋在睡的枕头内。又朱砂书符一道烧灰,暗暗搅茶内。若得夫主吃了茶,到晚夕睡了枕头,不过三日,自然有验。”妇人道:“请问先生,这四椿儿是怎的说?”贼瞎道:“好教娘子得知:用纱蒙眼,使夫主见你一似西施娇艳;用艾塞心,使他心爱到你;用针钉手,随你怎的不是,使他再不敢动手打你;用胶粘足者,使他再不往那里胡行。”妇人听言,满心欢喜。当下备了香烛纸马,替妇人烧了纸。到次日,使刘婆送了符水镇物与妇人,如法安顿停当,将符烧灰,顿下好茶,待的西门庆家来,妇人叫春梅递茶与他吃。到晚夕,与他共枕同床,过了一日两,两日三,似水如鱼,欢会异常。看观听说:但凡大小人家,师尼僧道,乳母牙婆,切记休招惹他,背地什么事不干出来?古人有四句格言说得好:

\[
堂前切莫走三婆,后门常锁莫通和。
院内有井防小口,便是祸少福星多。
\]
