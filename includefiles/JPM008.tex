%# -*- coding:utf-8 -*-
%%%%%%%%%%%%%%%%%%%%%%%%%%%%%%%%%%%%%%%%%%%%%%%%%%%%%%%%%%%%%%%%%%%%%%%%%%%%%%%%%%%%%


\chapter{盼情郎佳人占鬼卦\KG 烧夫灵和尚听淫声}


词曰:

\[
红曙卷窗纱,睡起半拖罗袂。何似等闲睡起,到日高还未。催花阵阵玉楼风,楼上人难睡。有了人儿一个,在眼前心里。
\]

话说西门庆自娶了玉楼在家,燕尔新婚,如胶似漆。又遇陈宅使文嫂儿来通信,六月十二日就要娶大姐过门。西门庆促忙促急攒造不出床来,就把孟玉楼陪来的一张南京描金彩漆拔步床陪了大姐。三朝九日,足乱了一个多月,不曾往潘金莲家去。把那妇人每日门儿倚遍,眼儿望穿。使王婆往他门首去寻,门首小厮知道是潘金莲使来的,多不理他。妇人盼的紧,见婆子回了,又叫小女儿街上去寻。那小妮子怎敢入他深宅大院?只在门首踅探,不见西门庆就回来了。来家被妇人哕骂在脸上,怪他没用,便要叫他跪着。饿到晌午,又不与他饭吃。此时正值三伏天道,妇人害热,分付迎儿热下水,伺候要洗澡。又做了一笼裹馅肉角儿,等西门庆来吃。身上只着薄纱短衫,坐在小凳上,盼不见西门庆到来,骂了几句负心贼。无情无绪,用纤手向脚上脱下两只红绣鞋儿来,试打一个相思卦。正是:

\[
逢人不敢高声语,暗卜金钱问远人。
\]
有《山坡羊》为证:

\[
凌波罗袜,天然生下,红云染就相思卦。似藕生芽,如莲卸花,怎生缠得些儿大!柳条儿比来刚半叉。他不念咱,咱何曾不念他!倚着门儿,私下帘儿,悄呀,空叫奴被儿里叫着他那名儿骂。你怎恋烟花,不来我家!奴眉儿淡淡教谁画?何处绿杨拴系马?他辜负咱,咱何曾辜负他!
\]
妇人打了一回相思卦,不觉困倦,就歪在床上盹睡着了。约一个时辰醒来,心中正没好气。迎儿问:“热了水,娘洗澡也不洗?”妇人就问:“角儿蒸熟了?拿来我看。”迎儿连忙拿到房中。妇人用纤手一数,原做下一扇笼三十个角儿,翻来复去只数得二十九个,便问:“那一个往那里去了?”迎儿道:“我并没看见,只怕娘错数了。”妇人道:“我亲数了两遍,三十个角儿,要等你爹来吃。你如何偷吃了一个?好娇态淫妇奴才,你害馋痨馋痞,心里要想这个角儿吃!你大碗小碗吃捣不下饭去,我做下孝顺你来!”便不由分说,把这小妮子跣剥去身上衣服,拿马鞭子打了二三十下,打的妮子杀猪般也似叫。问着他:“你不承认,我定打你百数!”打的妮子急了,说道:“娘休打,是我害饿的慌,偷吃了一个。”妇人道:“你偷了,如何赖我错数?眼看着就是个牢头祸根淫妇!有那亡八在时,轻学重告,今日往那里去了?还在我跟前弄神弄鬼!我只把你这牢头淫妇,打下你下截来!”打了一回,穿上小衣,放他起来,分付在旁打扇。打了一回扇,口中说道:“贼淫妇,你舒过脸来,等我掐你这皮脸两下子。”那妮子真个舒着脸,被妇人尖指甲掐了两道血口子,才饶了他。

良久,走到镜台前,从新妆点出来,门帘下站立。也是天假其便,只见玳安夹着毡包,骑着马,打妇人门首经过。妇人叫住,问他往何处去来。那小厮说话乖觉,常跟西门庆在妇人家行走,妇人常与他些浸润,以此滑熟。一面下马来,说道:“俺爹使我送人情,往守备府里去来。”妇人叫进门来,问道:“你爹家中有甚事,如何一向不来傍个影儿?想必另续上了一个心甜的姊妹了。”玳安道:“俺爹再没续上姊妹,只是这几日家中事忙,不得脱身来看六姨。”妇人道:“就是家中有事,那里丢我恁个半月,音信不送一个儿!只是不放在心儿上。”因问玳安:“有甚么事?你对我说。”那小厮嘻嘻只是笑,不肯说。妇人见玳安笑得有因,愈丁紧问道:“端的有甚事?”玳安笑道:“只说有椿事儿罢了,六姨只顾吹毛求疵问怎的?”妇人道:“好小油嘴儿,你不对我说,我就恼你一生。”小厮道:“我对六姨说,六姨休对爹说是我说的。”妇人道:“我决不对他说。”玳安就如此这般,把家中娶孟玉楼之事,从头至尾告诉了一遍。这妇人不听便罢,听了由不得珠泪儿顺着香腮流将下来。玳安慌了,便道:“六姨,你原来这等量窄,我故此不对你说。”妇人倚定门儿,长叹了一口气,说道:“玳安,你不知道,我与他从前以往那样恩情,今日如何一旦抛闪了。”止不住纷纷落下泪来。玳安道:“六姨,你何苦如此?家中俺娘也不管着他。”妇人便道:“玳安,你听告诉:乔才心邪,不来一月。奴绣鸳衾旷了三十夜。他俏心儿别,俺痴心儿呆,不合将人十分热。常言道容易得来容易舍。兴,过也;缘,分也。”说毕又哭。玳安道:“六姨,你休哭。俺爹怕不也只在这两日,他生日待来也。你写几个字儿,等我替你捎去,与俺爹看了,必然就来。”妇人道:“是必累你,请的他来。到明日,我做双好鞋与你穿。我这里也要等他来,与他上寿哩。他若不来,都在你小油嘴身上。”说毕,令迎儿把桌上蒸下的角儿,装了一碟,打发玳安儿吃茶。一面走入房中,取过一幅花笺,又轻拈玉管,款弄羊毛,须臾,写了一首《寄生草》。词曰:

\[
将奴这知心话,付花笺寄与他。想当初结下青丝发,门儿倚遍帘儿下,受了些没打弄的耽惊怕。你今果是负了奴心,不来还我香罗帕。
\]
写就,叠成一个方胜儿,封停当,付与玳安收了,道:“好歹多上覆他。待他生日,千万来走走。奴这里专望。”那玳安吃了点心,妇人又与数十文钱。临出门上马,妇人道:“你到家见你爹,就说六姨好不骂你。他若不来,你就说六姨到明日坐轿子亲自来哩。”玳安道:“六姨,自吃你卖粉团的撞见了敲板儿蛮子叫冤屈——麻饭胳胆的帐。”说毕,骑马去了。

那妇人每日长等短等,如石沉大海。七月将尽,到了他生辰。这妇人挨一日似三秋,盼一夜如半夏,等得杳无音信。不觉银牙暗咬,星眼流波。至晚,只得又叫王婆来,安排酒肉与他吃了,向头上拔下一根金头银簪子与他,央往西门庆家去请他来。王婆道:“这早晚,茶前酒后,他定也不来。待老身明日侵早请他去罢。”妇人道:“干娘,是必记心,休要忘了!”婆子道:“老身管着那一门儿,肯误了勾当?”这婆子非钱而不行,得了这根簪子,吃得脸红红,归家去了。且说妇人在房中,香薰鸳被,款剔银灯,睡不着,短叹长吁。正是:

\[
得多少琵琶夜久殷勤弄,寂寞空房不忍弹。
\]
于是独自弹着琵琶,唱一个《绵搭絮》:

\[
谁想你另有了裙钗,气的奴似醉如痴,斜倚定帏屏故意儿猜,不明白。怎生丢开?传书寄柬,你又不来。你若负了奴的恩情,人不为仇天降灾。
\]

妇人一夜翻来覆去,不曾睡着。巴到天明,就使迎儿:“过间壁瞧王奶奶请你爹去了不曾?”迎儿去不多时,说:“王奶奶老早就出去了。”

且说那婆子早晨出门,来到西门庆门首探问,都说不知道。在对门墙脚下等勾多时,只见傅伙计来开铺子。婆子走向前,道了万福:“动问一声,大官人在家么?”傅伙计道:“你老人家寻他怎的?早是问着我,第二个也不知他。大官人昨日寿诞,在家请客,吃了一日酒,到晚拉众朋友往院里去了,一夜通没回家。你往那里去寻他!”这婆子拜辞,出县前来到东街口,正往勾栏那条巷去。只见西门庆骑着马远远从东来,两个小厮跟随,此时宿酒未醒,醉眼摩娑,前合后仰。被婆子高声叫道:“大官人,少吃些儿怎的!”向前一把手把马嚼环扯住。西门庆醉中问道:“你是王干娘,你来想是六姐寻我?”那婆子向他耳畔低言。道不数句,西门庆道:“小厮来家对我说来,我知道六姐恼我哩,我如今就去。”那西门庆一面跟着他,两个一递一句,整说了一路话。

比及到妇人门首,婆子先入去,报道:“大娘子恭喜,还亏老身,没半个时辰,把大官人请将来了。”妇人听见他来,就象天上掉下来的一般,连忙出房来迎接。西门庆摇着扇儿进来,带酒半酣,与妇人唱喏。妇人还了万福,说道:“大官人,贵人稀见面!怎的把奴丢了,一向不来傍个影儿?家中新娘子陪伴,如胶似漆,那里想起奴家来!”西门庆道:“你休听人胡说,那讨什么新娘子来!因小女出嫁,忙了几日,不曾得闲工夫来看你。”妇人道:“你还哄我哩!你若不是怜新弃旧,另有别人,你指着旺跳身子说个誓,我方信你。”西门庆道:“我若负了你,生碗来大疔疮,害三五年黄病,匾担大蛆叮口袋。”妇人道:“负心的贼!匾担大蛆叮口袋,管你甚事?”一手向他头上把一顶新缨子瓦楞帽儿撮下来,望地上只一丢。慌的王婆地下拾起来,替他放在桌上,说道:“大娘子,只怪老身不去请大官人,来就是这般的。”妇人又向他头上拔下一根簪儿,拿在手里观看,却是一点油金簪儿,上面笈着两溜字儿:“金勒马嘶芳草地,玉楼人醉杏花天。”却是孟玉楼带来的。妇人猜做那个唱的送他的,夺了放在袖子里,说道:“你还不变心哩!奴与你的簪儿那里去了?”西门庆道:“你那根簪子,前日因酒醉跌下马来,把帽子落了,头发散开,寻时就不见了。”妇人将手在向西门庆脸边弹个响榧子,道:“哥哥儿,你醉的眼恁花了,哄三岁孩儿也不信!”王婆在傍插口道:“大娘子休怪!大官人,他离城四十里见蜜蜂儿刺屎,出门交獭象绊了一交,原来觑远不觑近。”西门庆道:“紧自他麻犯人,你又自作耍。”妇人见他手中拿着一把红骨细洒金、金钉铰川扇儿,取过来迎亮处只一照,原来妇人久惯知风月中事,见扇上多是牙咬的碎眼儿,就疑是那个妙人与他的。不由分说,两把折了。西门庆救时,已是扯的烂了,说道:“这扇子是我一个朋友卜志道送我的,一向藏着不曾用,今日才拿了三日,被你扯烂了。”

那妇人奚落了他一回,只见迎儿拿茶来,便叫迎儿放下茶托,与西门庆磕头。王婆道:“你两口子刮聒了这半日也勾了,休要误了勾当。老身厨下收拾去也。”妇人一边分付迎儿,将预先安排下与西门庆上寿的酒肴,整理停当,拿到房中,摆在桌上。妇人向箱中取出与西门庆上寿的物事,用盘盛着,摆在面前,与西门庆观看。却是一双玄色段子鞋;一双挑线香草边阑、松竹梅花岁寒三友酱色段子护膝;一条纱绿潞绸、水光绢里儿紫线带儿,里面装着排草玫瑰花兜肚;一根并头莲瓣簪儿。簪儿上笈着五言四句诗一首,云:“奴有并头莲,赠与君关髻。凡事同头上,切勿轻相弃。”西门庆一见满心欢喜,把妇人一手搂过,亲了个嘴,说道:“怎知你有如此聪慧!”妇人教迎儿执壶斟一杯与西门庆,花枝招扬,插烛也似磕了四个头。那西门庆连忙拖起来。两个并肩而坐,交杯换盏饮酒。那王婆陪着吃了几杯酒,吃的脸红红的,告辞回家去了。二人自在取乐玩耍。妇人陪伴西门庆饮酒多时,看看天色晚来,但见:

\[
密云迷晚岫,暗雾锁长空。群星与皓月争辉,绿水共青天同碧。僧投古寺,深林中嚷嚷鸦飞;客奔荒村,闾巷内汪汪犬吠。
\]

当下西门庆分付小厮回马家去,就在妇人家歇了。到晚夕,二人尽力盘桓,淫欲无度。

常言道:乐极生悲。光阴迅速,单表武松自领知县书礼驮担,离了清河县,竟到东京朱太尉处,下了书礼,交割了箱驮。等了几日,讨得回书,领一行人取路回山东而来。去时三四月天气,回来却淡暑新秋,路上雨水连绵,迟了日限。前后往回也有三个月光景。在路上行往坐卧,只觉得神思不安,身心恍惚,不免先差了一个土兵,预报与知县相公。又私自寄一封家书与他哥哥武大,说他只在八月内准还。那土兵先下了知县相公禀帖,然后迳来抓寻武大家。可可天假其便,王婆正在门首。那土兵见武大家门关着,才要叫门,婆子便问:“你是寻谁的?”土兵道:“我是武都头差来下书与他哥哥。”婆子道:“武大郎不在家,都上坟去了。你有书信,交与我,等他回来,我递与他,也是一般。”那土兵向前唱了一个喏,便向身边取出家书来交与王婆,忙忙骑上头口去了。

这王婆拿着那封书,从后门走过妇人家来。原来妇人和西门庆狂了半夜,约睡至饭时还不起来。王婆叫道:“大官人、娘子起来,和你们说话。如今武二差土兵寄书来与他哥哥,说他不久就到。我接下,打发他去了。你们不可迟滞,须要早作长便。”那西门庆不听万事皆休,听了此言,正是:分门八块顶梁骨,倾下半桶冰雪来。慌忙与妇人都起来,穿上衣服,请王婆到房内坐下。取出书来与西门庆看。书中写着,不过中秋回家。二人都慌了手脚,说道:“如此怎了?干娘遮藏我每则个,恩有重报,不敢有忘。我如今二人情深似海,不能相舍。武二那厮回来,便要分散,如何是好?”婆子道:“大官人,有什么难处之事!我前日已说过,幼嫁由亲,后嫁由身。古来叔嫂不通门户,如今武大已百日来到,大娘子请上几个和尚,把这灵牌子烧了。趁武二未到家,大官人一顶轿子娶了家去。等武二那厮回来,我自有话说。他敢怎的?自此你二人自在一生,岂不是妙!”西门庆便道:“干娘说的是。”当日西门庆和妇人用毕早饭,约定八月初六日,是武大百日,请僧烧灵。初八日晚,娶妇人家去。三人计议已定。不一时,玳安拿马来接回家,不在话下。

光阴似箭,日月如梭,又早到了八月初六日。西门庆拿了数两碎银钱,来妇人家,教王婆报恩寺请了六个僧,在家做水陆,超度武大,晚夕除灵。道人头五更就挑了经担来,铺陈道场,悬挂佛像。王婆伴厨子在灶上安排斋供。西门庆那日就在妇人家歇了。不一时,和尚来到,摇响灵杵,打动鼓钹,讽诵经忏,宣扬法事,不必细说。

且说潘金莲怎肯斋戒,陪伴西门庆睡到日头半天,还不起来。和尚请斋主拈香佥字,证盟礼佛,妇人方才起来梳洗,乔素打扮,来到佛前参拜。众和尚见了武大这老婆,一个个都迷了佛性禅心,关不住心猿意马,七颠八倒,酥成一块。但见:

\[
班首轻狂,念佛号不知颠倒;维摩昏乱,诵经言岂顾高低。烧香行者,推倒花瓶;秉烛头陀,误拿香盒。宣盟表白,大宋国错称做大唐国;忏罪阇黎,武大郎几念武大娘。长老心忙,打鼓借拿徒弟手;沙弥情荡,罄槌敲破老僧头。从前苦行一时休,万个金刚降不住。
\]
妇人在佛前烧了香,佥了字,拜礼佛毕,回房去依旧陪伴西门庆。摆上酒席荤腥,自去取乐。西门庆分付王婆:“有事你自答应便了,休教他来聒噪六姐。”婆子哈哈笑道:“你两口儿只管受用,由着老娘和那秃厮缠。”

且说从和尚见了武大老婆乔模乔样,多记在心里。到午斋往寺中歇晌回来,妇人正和西门庆在房里饮酒作欢。原来妇人卧房与佛堂止隔一道板壁。有一个僧人先到,走在妇人窗下水盆里洗手,忽听见妇人在房里颤声柔气,呻呻吟吟,哼哼唧唧,恰似有人交媾一般。遂推洗手,立住脚听。只听得妇人口里喘声呼叫:“达达,你只顾搧打到几时?只怕和尚来听见。饶了奴,快些丢了罢!”西门庆道:“你且休慌!我还要在盖子上烧一下儿哩!”不想都被这秃厮听了个不亦乐乎。落后众和尚到齐了,吹打起法事来,一个传一个,都知妇人有汉子在屋里,不觉都手之舞之,足之蹈之。临佛事完满,晚夕送灵化财出去,妇人又早除了孝髻,登时把灵牌并佛烧了。那贼秃冷眼瞧见,帘子里一个汉子和婆娘影影绰绰并肩站着,想起白日里听见那些勾当,只顾乱打鼓搧钹不住。被风把长老的僧伽帽刮在地上,露出青旋旋光头,不去拾,只顾搧钹打鼓,笑成一块。王婆便叫道:“师父,纸马已烧过了,还只顾搧打怎的?”和尚答道:“还有纸炉盖子上没烧过。”西门庆听见,一面令王婆快打发衬钱与他。长老道:“请斋主娘子谢谢。”妇人道:“干娘说免了罢。”众和尚道:“不如饶了罢。”一齐笑的去了。正是:隔墙须有耳,窗外岂无人!有诗为证:

\[
淫妇烧灵志不平,阇黎窃壁听淫声。
果然佛法能消罪,亡者闻之亦惨魂。
\]
