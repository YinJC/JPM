%# -*- coding:utf-8 -*-
%%%%%%%%%%%%%%%%%%%%%%%%%%%%%%%%%%%%%%%%%%%%%%%%%%%%%%%%%%%%%%%%%%%%%%%%%%%%%%%%%%%%%


\chapter{玉箫跪受三章约\KG 书童私挂一帆风}


诗曰:

\[
玉殒珠沉思悄然,明中流泪暗相怜。
常图蛱蝶花楼下,记效鸳鸯翠幕前。
只有梦魂能结雨,更无心绪学非烟。
朱颜皓齿归黄土,脉脉空寻再世缘。
\]

话说众人散了,已有鸡唱时分,西门庆歇息去了。玳安拿了一大壶酒、几碟下饭,在铺子里还要和傅伙计、陈敬济同吃。傅伙计老头子熬到这咱,已是坐不住,搭下铺就倒在炕上,向玳安道:“你自和平安吃罢,陈姐夫想也不来了。”玳安叫进平安来,两个把那酒你一钟我一盏都吃了。收过家伙,平安便去门房里睡了。玳安一面关上铺子门,上炕和傅伙计两个对厮脚儿睡下。傅伙计因闲话,向玳安说道:“你六娘没了,这等棺椁念经发送,也够他了。”玳安道:“他的福好,只是不长寿。俺爹饶使了这些钱,还使不着俺爹的哩。俺六娘嫁俺爹,瞒不过你老人家,他带了多少带头来!别人不知道,我知道。银子休说,只金珠玩好、玉带、绦环、\textuni{4BFC}髻、值钱的宝石,也不知有多少。为甚俺爹心里疼?不是疼人,是疼钱。若说起六娘的性格儿,一家子都不如他,又谦让又和气,见了人,只是一面儿笑,自来也不曾喝俺每一喝,并没失口骂俺每一句‘奴才’。使俺每买东西,只拈块儿。俺每但说:‘娘,拿等子,你称称。’他便笑道:‘拿去罢,称什么。你不图落图什么来?只要替我买值着。’这一家子,那个不借他银使?只有借出来,没有个还进去的。还也罢,不还也罢。俺大娘和俺三娘使钱也好。只是五娘和二娘,悭吝的紧。他当家,俺每就遭瘟来。会胜买东西,也不与你个足数,绑着鬼,一钱银子,只称九分半,着紧只九分,俺每莫不赔出来!”傅伙计道:“就是你大娘还好些。”玳安道:“虽故俺大娘好,毛司火性儿,一回家好,娘儿每亲亲哒哒说话儿,你只休恼着他,不论谁,他也骂你几句儿。总不如六娘,万人无怨,又常在爹跟前替俺每说方便儿。随问天来大事,俺每央他央儿对爹说,无有个不依。只是五娘,行动就说:‘你看我对爹说不说!’把这打只提在口里。如今春梅姐,又是个合气星。——天生的都在他一屋里。”傅伙计道:“你五娘来这里也好几年了。”玳安道:“你老人家是知道的,想的起他那咱来的光景哩。他一个亲娘也不认的,来一遭,要便抢的哭了家去。如今六娘死了,这前边又是他的世界,明日那个管打扫花园,干净不干净,还吃他骂的狗血喷了头哩!”两个说了一回,那傅伙计在枕上齁齁就睡着了。玳安亦有酒了,合上眼,不知天高地下,直至红日三竿,都还未起来。

原来西门庆每常在前边灵前睡,早晨玉箫出来收叠床铺,西门庆便往后边梳头去。书童蓬着头,要便和他两个在前边打牙犯嘴,互相嘲逗,半日才进后边去。不想这日西门庆归上房歇去,玉箫赶人没起来,暗暗走出来,与书童约了,走在花园书房里干营生去了。不料潘金莲起的早,蓦地走到厅上,只见灵前灯儿也没了,大棚里丢的桌椅横三竖四,没一个人儿,只有画童儿在那里扫地。金莲道:“贼囚根子,干净只你在这里,都往那里去了?”画童道:“他每都还没起来哩。”金莲道:“你且丢下笤帚,到前边对你姐夫说,有白绢拿一匹来,你潘姥姥还少一条孝裙子,再拿一副头须系腰来与他。他今日家去。”画童道:“怕不俺姐夫还睡哩,等我问他去。”良久回来道:“姐夫说不是他的首尾,书童哥与崔本哥管孝帐。娘问书童哥要就是了。”金莲道:“知道那奴才往那去了,你去寻他来。”画童向厢房里瞧了瞧,说道:“才在这里来,敢往花园书房里梳头去了。”金莲说道:“你自扫地,等我自家问这囚根子要去。”因走到花园书房内,忽然听见里面有人笑声。推开门,只见书童和玉箫在床上正干得好哩。便骂道:“好囚根子,你两个干得好事!”唬得两个做手脚不迭,齐跪在地下哀告。金莲道:“贼囚根子,你且拿一匹孝绢、一匹布来,打发你潘姥姥家去着。”书童连忙拿来递上。金莲迳归房来。

那玉箫跟到房中,打旋磨儿跪在地下央及:“五娘,千万休对爹说。”金莲便问:“贼狗肉,你和我实说,从前已往,偷了几遭?一字儿休瞒我,便罢。”那玉箫便把和他偷的缘由说了一遍。金莲道:“既要我饶你,你要依我三件事。”玉箫道:“娘饶了我,随问几件事我也依娘。”金莲道:“第一件,你娘房里,但凡大小事儿,就来告我说。你不说,我打听出来,定不饶你。第二件,我但问你要甚么,你就捎出来与我。第三件,你娘向来没有身孕,如今他怎生便有了?”玉箫道:“不瞒五娘说,俺娘如此这般,吃了薛姑子的衣胞符药,便有了。”潘金莲一一听记在心,才不对西门庆说了。

书童见潘金莲冷笑领进玉箫去了,知此事有几分不谐。向书房厨柜内收拾了许多手帕汗巾、挑牙簪纽,并收的人情,他自己也攒有十来两银子,又到前边柜上诓了傅伙计二十两,只说要买孝绢,迳出城外,雇了长行头口,到码头上,搭在乡里船上,往苏州原籍家去了。正是:

\[
撞碎玉笼飞彩凤,顿开金锁走蛟龙。
\]

那日,李桂姐、吴银儿、郑爱月都要家去了。薛内相、刘内相早晨差人抬三牲桌面来祭奠烧纸。又每人送了一两银子伴宿分资,叫了两个唱道情的来,白日里要和西门庆坐坐。紧等着要打发孝绢,寻书童儿要钥匙,一地里寻不着。傅伙计道:“他早晨问我柜上要了二十两银子买孝绢去了,口称爹吩咐他孝绢不够,敢是向门外买去了?”西门庆道:“我并没吩咐他,如何问你要银子?”一面使人往门外绢铺找寻,那里得来!月娘向西门庆说:“我猜这奴才有些跷蹊,不知弄下甚么硶儿,拐了几两银子走了。你那书房里还大瞧瞧,只怕还拿甚么去了。”西门庆走到两个书房里都瞧了,只见库房里钥匙挂在墙上,大橱柜里不见了许多汗巾手帕,并书礼银子、挑牙纽扣之类,西门庆心中大怒,叫将该地方管役来,吩咐:“各处三街两巷与我访缉。”那里得来!正是:

\[
不独怀家归兴急,五湖烟水正茫茫。
\]

那日,薛内相从晌午就坐轿来了。西门庆请下吴大舅、应伯爵、温秀才相陪。先到灵前上香,打了个问讯,然后与西门庆叙礼,说道:“可伤,可伤!如夫人是甚病儿殁了?”西门庆道:“不幸患崩泻之疾殁了,多谢老公公费心。”薛内相道:“没多儿,将就表意罢了。”因看见挂的影,说道:“好位标致娘子!正好青春享福,只是去世太早些。”温秀才在旁道:“物之不齐,物之情也。穷通寿夭,自有个定数,虽圣人亦不能强。”薛内相扭回头来,见温秀才穿着衣巾,因说道:“此位老先儿是那学里的?”温秀才躬身道:“学生不才,备名府庠。”薛内相道:“我瞧瞧娘子的棺木儿。”西门庆即令左右把两边帐子撩起,薛内相进去观看了一遍,极口称赞道:“好副板儿!请问多少价买的?”西门庆道:“也是舍亲的一副板,学生回了他的来了。”应伯爵道:“请老公公试估估,那里地道,甚么名色?”薛内相仔细看了说:“此板不是建昌,就是副镇远。”伯爵道:“就是镇远,也值不多。”薛内相道:“最高者,必定是杨宣榆。”伯爵道:“杨宣榆单薄短小,怎么看得过!此板还在杨宣榆之上,名唤做桃花洞,在于湖广武陵川中。昔日唐渔父入此洞中,曾见秦时毛女在此避兵,是个人迹罕到之处。此板七尺多长,四寸厚,二尺五宽。还看一半亲家分上,还要了三百七十两银子哩。公公,你不曾看见,解开喷鼻香的,里外俱有花色。”薛内相道:“是娘子这等大福,才享用了这板。俺每内官家,到明日死了,还没有这等发送哩。”吴大舅道:“老公公好说,与朝廷有分的人,享大爵禄,俺们外官焉能赶的上。老公公日近清光,代万岁传宣金口。见今童老爷加封王爵,子孙皆服蟒腰玉,何所不至哉!”薛内相便道:“此位会说话的兄,请问上姓?”西门庆道:“此是妻兄吴大哥,见居本卫千户之职。”薛内相道:“就是此位娘子令兄么?”西门庆道:“不是。乃贱荆之兄。”薛内相复于吴大舅声诺说道:“吴大人,失瞻!”

看了一回,西门庆让至卷棚内,正面安放一把交椅,薛内相坐下,打茶的拿上茶来吃了。薛内相道:“刘公公怎的这咱还不到?叫我答应的迎迎去。”青衣人跪下禀道:“小的邀刘公公去来,刘公公轿已伺候下了,便来也。”薛内相又问道:“那两个唱道情的来了不曾?”西门庆道:“早上就来了。——叫上来!”不一时,走来面前磕头。薛内相道:“你每吃了饭不曾?”那人道:“小的每吃了饭了。”薛内相道:“既吃了饭,你每今日用心答应,我重赏你。”西门庆道:“老公公,学生这里还预备着一起戏子,唱与老公公听。”薛内相问:“是那里戏子?”西门庆道:“是一班海盐戏子。”薛内相道:“那蛮声哈剌,谁晓的他唱的是甚么!那酸子每在寒窗之下,三年受苦,九载遨游,背着琴剑书箱来京应举,得了个官,又无妻小在身边,便希罕他这样人。你我一个光身汉、老内相,要他做甚么?”温秀才在旁边笑说道:“老公公说话,太不近情了。居之齐则齐声,居之楚则楚声。老公公处于高堂广厦,岂无一动其心哉?”这薛内相便拍手笑将起来道:“我就忘了温先儿在这里。你每外官,原来只护外官。”温秀才道:“虽是士大夫,也只是秀才做的。老公公砍一枝损百林,兔死狐悲,物伤其类。”薛内相道:“不然。一方之地,有贤有愚。”

正说着,忽左右来报:“刘公公下轿了。”吴大舅等出去迎接进来,向灵前作了揖。叙礼已毕,薛内相道:“刘公公,你怎的这咱才来?”刘内相道:“北边徐同家来拜望,陪他坐了一回,打发去了。”一面分席坐下,左右递茶上去。因问答应的:“祭奠桌面儿都摆上了不曾?”下边人说:“都排停当了。”刘内相道:“咱每去烧了纸罢。”西门庆道:“老公公不消多礼,头里已是见过礼了。”刘内相道:“此来为何?还当亲祭祭。”当下,左右捧过香来,两个内相上了香,递了三钟酒,拜下去。西门庆道:“老公公请起。”于是拜了两拜起来,西门庆还了礼,复至卷棚内坐下。然后收拾安席,递酒上坐。两位内相分左右坐了,吴大舅、温秀才、应伯爵从次,西门庆下边相陪。子弟鼓板响动,递了关目揭帖。两位内相看了一回,拣了一段《刘智远白兔记》。唱了还未几折,心下不耐烦,一面叫上两个唱道情的去,打起渔鼓,并肩朝上,高声唱了一套“韩文公雪拥蓝关”故事下去。

薛内相便与刘内相两个说说话儿,道:“刘哥,你不知道,昨日这八月初十日,下大雨如注,雷电把内里凝神殿上鸱尾裘碎了,唬死了许多宫人。朝廷大惧,命各官修省,逐日在上清宫宣《精灵疏》建醮。禁屠十日,法司停刑,百官不许奏事。昨日大金遣使臣进表,要割内地三镇,依着蔡京那老贼,就要许他。掣童掌事的兵马,交都御史谭积、黄安十大使节制三边兵马,又不肯,还交多官计议。昨日立冬,万岁出来祭太庙,太常寺一员博士,名唤方轸,早晨打扫,看见太庙砖缝出血,殿东北上地陷了一角,写表奏知万岁。科道官上本,极言童掌事大了,宦官不可封王。如今马上差官,拿金牌去取童掌事回京。”刘内相道:“你我如今出来在外做土官,那朝事也不干咱每。俗语道,咱过了一日是一日。便塌了天,还有四个大汉。到明天,大宋江山管情被这些酸子弄坏了。王十九,咱每只吃酒!”因叫唱道情的上来,吩咐:“你唱个‘李白好贪杯’的故事。”那人立在席前,打动渔鼓,又唱了一回。

直吃至日暮时分,吩咐下人,看轿起身。西门庆款留不住,送出大门,喝道而去。回来,吩咐点起烛来,把桌席休动,留下吴大舅、应伯爵、温秀才坐的,又使小厮请傅伙计、甘伙计、韩道国、贲第传、崔本和陈敬济复坐。叫上子弟来吩咐:“还找着昨日《玉环记》上来。”因向伯爵道:“内相家不晓的南戏滋味。早知他不听,我今日不留他。”伯爵道:“哥,到辜负你的意思。内臣斜局的营生,他只喜《蓝关记》、捣喇小子山歌野调,那里晓的大关目悲欢离合!”于是下边打动鼓板,将昨日《玉环记》做不完的折数,一一紧做慢唱,都搬演出来。西门庆令小厮席上频斟美酒。伯爵与西门庆同桌而坐,便问:“他姐儿三个还没家去,怎的不叫出来递杯酒儿?”西门庆道:“你还想那一梦儿,他每去的不耐烦了!”伯爵道:“他每在这里住了有两三日?”西门庆道:“吴银儿住的久了。”当日,众人坐到三更时分,搬戏已完,方起身各散。西门庆邀下吴大舅,明日早些来陪上祭官员。与了戏子四两银子,打发出门。

到次日,周守备、荆都监、张团练、夏提刑,合卫许多官员,都合了分资,办了一副猪羊吃桌祭奠,有礼生读祝。西门庆预备酒席,李铭等三个小优儿伺候答应。到晌午,只听鼓响,祭礼到了。吴大舅、应伯爵、温秀才在门首迎接,只见后拥前呼,众官员下马,在前厅换衣服。良久,把祭品摆下,众官齐到灵前,西门庆与陈敬济还礼。礼生喝礼,三献毕,跪在旁边读祝,祭毕。西门庆下来谢礼已毕,吴大舅等让众官至卷棚内,宽去素服,待毕茶,就安席上坐,觥筹交错,殷勤劝酒。李铭等三个小优儿,银筝檀板,朝上弹唱。众官欢饮,直到日暮方散。西门庆还要留吴大舅众人坐,吴大舅道:“各人连日打搅,姐夫也辛苦了,各自歇息去罢。”当时告辞回家。正是:

\[
天上碧桃和露种,日边红杏倚云栽。
家中巨富人趋附,手内多时莫论财。
\]
