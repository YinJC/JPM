%# -*- coding:utf-8 -*-
%%%%%%%%%%%%%%%%%%%%%%%%%%%%%%%%%%%%%%%%%%%%%%%%%%%%%%%%%%%%%%%%%%%%%%%%%%%%%%%%%%%%%


\chapter{献芳樽内室乞恩\KG 受私贿后庭说事}


词曰:

\[
成吴越,怎禁他巧言相斗谍。平白地送暖偷寒,平白地送暖偷寒,猛可的搬唇弄舌。水晶丸不住撇,蘸刚锹一味撅。
\]

话说韩道国走到家门首打听,见浑家和兄弟韩二拴在铺中去了,急急走到铺子内,和来保计议。来保说:“你还早央应二叔来,对当家的说了,拿个帖儿对县中李老爹一说,不论多大事情都了了。”这韩道国竟到应怕爵家。他娘子儿使丫头出来回:“没人在家,不知往那里去了。只怕在西门大老爹家。”韩道国道:“没在他宅里。”问应宝,也跟出去了。韩道国慌了,往勾栏院里抓寻。原来伯爵被湖州何蛮子的兄弟何二蛮子——号叫何两峰,请在四条巷内何金蝉儿家吃酒。被韩道国抓着了,请出来。伯爵吃的脸红红的,帽檐上插着剔牙杖儿。韩道国唱了喏,拉到僻静处,如此这般告他说。伯爵道:“既有此事,我少不得陪你去。”于是辞了何两峰,与道国先同到家,问了端的。道国央及道:“此事明日只怕要解到县里去,只望二叔往大官府宅里说说,讨个帖儿,转与李老爹,求他只不教你侄妇见官。事毕重谢二叔。”说着跪在地下。伯爵用手拉起来,说道:“贤契,这些事儿,我不替你处?你快写个说帖,把一切闲话都丢开,只说你常不在家,被街坊这伙光棍时常打砖掠瓦,欺负娘子。你兄弟韩二气忿不过,和他嚷乱,反被这伙人群住,揪采踢打,同拴在铺里。望大官府发个帖儿,对李老爹说,只不教你令正出官,管情见个分上就是了。”那韩道国取笔砚,连忙写了说帖,安放袖中。

伯爵领他迳到西门庆门首,问守门的平安儿:“爹在家?”平安道:“爹在花园书房里。二爹和韩大叔请进去。”那应伯爵狗也不咬,走熟了的,同韩道国进入仪门,转过大厅,由鹿顶钻山进去,就是花园角门。抹过木香棚,三间小卷棚,名唤翡翠轩,乃西门庆夏月纳凉之所。前后帘拢掩映,四面花竹阴森,里面一明两暗书房。有画童儿小厮在那里扫地,说:“应二爹和韩大叔来了!”二人掀开帘子。进入明间内,书童看见便道:“请坐。俺爹刚才进后边去了。”一面使画童儿请去。画童儿走到后边金莲房内,问:“春梅姐,爹在这里?”春梅骂道:“贼见鬼小奴才儿!爹在间壁六娘房里不是,巴巴的跑来这里问!”画童便走过这边,只见绣春在石台基上坐的,悄悄问:“爹在房里?应二爹和韩大叔来了,在书房里等爹说话。”绣春道:“爹在房里,看着娘与哥裁衣服哩。”原来西门庆拿出口匹尺头来,一匹大红纻丝,一匹鹦哥绿潞绸,教李瓶儿替官哥裁毛衫、披袄、背心、护顶之类。在炕上正铺着大红毡条。奶子抱着哥儿,迎春执着熨斗。只见绣春进来,悄悄拉迎春一把,迎春道:“你拉我怎么的?拉撇了这火落在毡条上。”李瓶儿便问:“你平白拉他怎的?”绣春道:“画童说应二爹来了,请爹说话。”李瓶儿道:“小奴才儿,应二爹来,你进来说就是了,巴巴的扯他!”

西门庆分咐画童:“请二爹坐坐,我就来。”于是看裁完了衣服,便衣出来,书房内见伯爵二人,作揖坐下,韩道国打横。吃了茶,伯爵就开言说道:“韩大哥,你有甚话,对你大官府说。”西门庆道:“你有甚话说来。”韩道国才待说“街坊有伙不知姓名棍徒……”,被应伯爵拦住便道:“贤侄,你不是这等说了。噙着骨秃露着肉,也不是事。对着你家大官府在这里,越发打开后门说了罢:韩大哥常在铺子里上宿,家下没人,止是他娘子儿一人,还有个孩儿。左右街坊,有几个不三不四的人,见无人在家,时常打砖掠瓦鬼混。欺负的急了,他令弟韩二哥看不过,来家骂了几句,被这起光棍不由分说,群住了打个臭死。如今部拴在铺里,明早要解了往本县李大人那里去。他哭哭啼啼,央烦我来对哥说,讨个帖儿,对李大人说说,青目一二。有了他令弟也是一般,只不要他令正出官就是了。”因说:“你把那说帖儿拿出来与你大官人瞧,好差人替你去。”韩道国便向袖中取出,连忙双膝跪下,说道:“小人忝在老爹门下,万乞老爹看应二叔分上,俯就一二,举家没齿难忘。”西门庆一把手拉起,说道:“你请起来。”于是观看帖儿,上面写着:“犯妇王氏,乞青目免提。”西门庆道:“这帖子不是这等写了!只有你令弟韩二一人就是了。”向伯爵道:“比时我拿帖对县里说,不如只分咐地方改了报单,明日带来我衙门里来发落就是了。”伯爵教:“韩大哥,你还与恩老爹下个礼儿。这等亦发好了!”那韩道国又倒身磕头下去。西门庆教玳安:“你外边快叫个答应的班头来。”不一时,叫了个穿青衣的节级来,在旁边伺候。西门庆叫近前,分咐:“你去牛皮街韩伙计住处,问是那牌那铺地方,对那保甲说,就称是我的钧语,分咐把王氏即时与我放了。查出那几个光棍名字来,改了报帖,明日早解提刑院,我衙门里听审。”那节级应诺,领了言语出门。伯爵道:“韩大哥,你即一同跟了他,干你的事去罢,我还和大官人说话哩。”那韩道国千恩万谢出门,与节级同往牛皮街干事去了。

西门庆陪伯爵在翡翠轩坐下,因令玳安放桌儿:“你去对你大娘说,昨日砖厂刘公公送的木樨荷花酒,打开筛了来,我和应二叔吃,就把糟鲥鱼蒸了来。”伯爵举手道:“我还没谢的哥,昨日蒙哥送了那两尾好鲫鱼与我。送了一尾与家兄去,剩下一尾,对房下说,拿刀儿劈开,送了一段与小女,余者打成窄窄的块儿,拿他原旧红糟儿培着,再搅些香油,安放在一个磁罐内,留着我一早一晚吃饭儿,或遇有个人客儿来,蒸恁一碟儿上去,也不枉辜负了哥的盛情。”西门庆告诉:“刘太监的兄弟刘百户,因在河下管芦苇场,赚了几两银子,新买了一所庄子在五里店,拿皇木盖房,近日被我衙门里办事官缉听着,首了。依着夏龙溪,饶受他一百两银子,还要动本参送,申行省院。刘太监慌了,亲自拿着一百两银子到我这里,再三央及,只要事了。不瞒你说,咱家做着些薄生意,料也过了日子,那里希罕他这样钱!况刘太监平日与我相交,时常受他些礼,今日因这些事情,就又薄了面皮?教我丝毫没受他的,只教他将房屋连夜拆了。到衙门里,只打了他家人刘三二十,就发落开了。事毕,刘太监感情不过,宰了一口猪,送我一坛自造荷花酒,两包糟鲥鱼,重四十斤,又两匹妆花织金缎子,亲自来谢。彼此有光,见个情分。”伯爵道:“哥,你是希罕这个钱的?夏大人他出身行伍,起根立地上没有,他不挝些儿,拿甚过日?哥,你自从到任以来,也和他问了几桩事儿?”西门庆道:“大小也问了几件公事。别的到也罢了,只吃了他贪滥蹋婪,有事不论青红皂白,得了钱在手里就放了,成甚么道理!我便再三扭着不肯,‘你我虽是个武职官儿,掌着这刑条,还放些体面才好。’”说未了,酒菜齐至。西门庆将小金菊花杯斟荷花酒,陪伯爵吃。

不说两个说话儿,坐更余方散。且说那伙人,见青衣节级下地方,把妇人王氏放回家去,又拘总甲,查了各人名字,明早解提刑院问理,都各人口面相觑。就知韩道国是西门庆家伙计,寻的本家攊子,只落下韩二一人在铺里。都说这事弄的不好了。这韩道国又送了节级五钱银子,登时间保甲查写那几个名字,送到西门庆宅内,单等次日早解。

过一日,西门庆与夏提刑两位官,到衙门里坐厅。该地方保甲带上人去,头一起就是韩二,跪在头里。夏提刑先看报单:“牛皮街一牌四铺总甲萧成,为地方喧闹事……”第一个就叫韩二,第二个车淡,第三个管世宽,第四个游守,第三个郝贤。都叫过花名去。然后问韩二:“为什么起来?”那韩二先告道:“小的哥是买卖人,常不在家住的,小男幼女,被街坊这几个光棍,要便弹打胡博词儿,坐在门首,胡歌野调,夜晚打砖,百般欺负。小的在外另住,来哥家看视,含忍不过,骂了几句。被这伙棍徒,不由分说,揪倒在地,乱行踢打,获在老爷案下。望老爷查情。”夏提刑便问:“你怎么说?”那伙人一齐告道:“老爷休信他巧对!他是耍钱的捣鬼。他哥不在家,和他嫂子王氏有奸。王氏平日倚逞刁泼毁驾街坊。昨日被小的们捉住,见有底衣为证。”夏提刑因问保甲萧成:“那王氏怎的不见?”萧成怎的好回节级放了?只说:“王氏脚小,路上走不动,便来。”那韩二在下边,两只眼只看着西门庆。良久,西门庆欠身望夏提刑道:“长官也不消要这王氏。想必王氏有些姿色,这光棍来调戏他不遂,捏成这个圈套。”因叫那为首的车淡上去,问道:“你在那里捉住那韩二来?”众人道:“昨日在他屋里捉来。”又问韩二:“王氏是你甚么人?”保甲道:“是他嫂子儿。”又问保甲:“这伙人打那里进他屋里?”保甲道:“越墙进去。”西门庆大怒,骂道:“我把你这起光棍!他既是小叔,王氏也是有服之亲,莫不不许上门行走?相你这起光棍,你是他什么人,如何敢越墙进去?况他家男子不在,又有幼女在房中,非奸即盗了。”喝令左右拿夹棍来,每人一夹、二十大棍,打的皮开肉绽,鲜血迸流。况四五个都是少年子弟,出娘胞胎未经刑杖,一个个打的号哭动天,呻吟满地。这西门庆也不等夏提刑开口,分咐:“韩二出去听候。把四个都与我收监,不日取供送问。”四人到监中都互相抱怨,个个都怀鬼胎。监中人都吓恐他:“你四个若送问,都是徒罪。到了外府州县,皆是死数。”这些人慌了,等的家下人来送饭,捎信出去,教各人父兄使钱,上下寻人情。内中有拿人情央及夏提刑,夏提刑说:“这王氏的丈夫是你西门老爹门下的伙计。他在中间扭着要送问,同僚上,我又不好处得。你须还寻人情和他说去。”也有央吴大舅出来说的。人都知西门庆家有钱,不敢来打点。

四家父兄都慌了,会在一处。内中一个说道:“也不消再央吴千户,他也不依。我闻得人说,东街上住的开绸绢铺应大哥兄弟应二,和他契厚。咱不如凑了几十两银子,封与应二,教他替咱们说说,管情极好。”于是车淡的父亲开酒店的车老儿为首,每人拿十两银子来,共凑了四十两银子,齐到应伯爵家,央他对西门庆说。伯爵收下,打发众人去了。他娘子儿便说:“你既替韩伙计出力,摆布这起人,如何又揽下这银子,反替他说方便,不惹韩伙计怪?”伯爵道:“我可知不好说的。我别自有处。”因把银子兑了十五两,包放袖中,早到西门庆家。西门庆还未回来。伯爵进厅上,只见书童正从西厢房书房内出来,头带瓦楞帽儿,撇着金头莲瓣簪子,身上穿着苏州绢直掇,玉色纱\textuni{2773D}儿,凉鞋净袜。说道:“二爹请客位内坐。”交画童儿后边拿茶去,说道:“小厮,我使你拿茶与应二爹,你不动,且耍子儿。等爹来家,看我说不说!”那小厮就拿茶去了。伯爵便问:“你爹衙门里还没来家?”书童道:“刚才答应的来,说爹衙门散了,和夏老爹门外拜客去了。二爹有甚话说?”伯爵道:“没甚话。”书童道:“二爹前日说的韩伙计那事,爹昨日到衙门里,把那伙人都打了收监,明日做文书还要送问他。”伯爵拉他到僻静处,和他说:“如今又一件,那伙人家属如此这般,听见要送问,都害怕了。昨日晚夕,到我家哭哭啼啼,再三跪着央及我,教对你爹说。我想我已是替韩伙计说在先,怎又好管他的,惹的韩伙计不怪?没奈何,教他四家处了这十五两银子,看你取巧对你爹说,看怎么将就饶他放了罢。”因向袖中取出银子来递与书童。书童打开看了,大小四锭零四块。说道:“既是应二爹分上,交他再拿五两来,待小的替他说,还不知爹肯不肯。昨日吴大舅亲自来和爹说了,爹不依。小的虼蚤脸儿——好大面皮!实对二爹说,小的这银子,不独自一个使,还破些钞儿,转达知俺生哥的六娘,绕个弯儿替他说,才了他此事。”伯爵道:“既如此,等我和他说。你好歹替他上心些,他后晌些来讨回话。”书童道:“爹不知多早来家,你教他明日早来罢。”说毕,伯爵去了。

这书童把银子拿到铺子,镏下一两五钱来,教人买了一坛金华酒,两只烧鸭,两只鸡,一钱银子鲜鱼,一肘蹄子,二钱顶皮酥果馅饼儿,一钱银子的搽穰卷儿,送到来兴儿屋里,央及他媳妇惠秀替他整理,安排端正。那一日,潘金莲不在家,从早间就坐轿子往门外潘姥姥家做生日去了。书童使画童儿用方盒把下饭先拿在李瓶儿房中,然后又提了一坛金华酒进去。李瓶儿便问:“是那里的?”画童道:“是书童哥送来孝顺娘的。”李瓶儿笑道:“贼囚!他怎的孝顺我?”良久,书童儿进来,见瓶儿在描金炕床上,引着玳瑁猫儿和哥儿耍子。因说道:“贼囚!你送了这些东西来与谁吃,”那书童只是笑。李瓶儿道:“你不言语,笑是怎的说?”书童道:“小的不孝顺娘,再孝顺谁!”李瓶儿道:“贼囚!你平白好好的,怎么孝顺我?你不说明白,我也不吃。”那书童把酒打开,菜蔬都摆在小桌上,教迎春取了把银素筛了来,倾酒在锺内,双手递上去,跪下说道:“娘吃过,等小的对娘说。”李瓶儿道:“你有甚事,说了我才吃。不说,你就跪一百年,我也是不吃。”又道:“你起来说。”那书童于是把应伯爵所央四人之事,从头诉说一遍:“他先替韩伙计说了,不好来说得,央及小的先来禀过娘。等爹问,休说是小的说,只假做花大舅那头使人来说。小的写下个帖儿在前边书房内,只说是娘递与小的,教与爹看。娘再加一美言。况昨日衙门里爹已是打过他,爹胡乱做个处断,放了他罢,也是老大的阴骘。”李瓶儿笑道:“原来也是这个事!不打紧,等你爹来家,我和他说就是了。你平白整治这些东西来做什么?”又道:“贼囚!你想必问他起发些东西了,”书童道:“不瞒娘说,他送了小的五两银子。”李瓶儿道:“贼囚!你倒且是会排铺赚钱!”于是不吃小锺,旋教迎春取了个大银衢花杯来,先吃了两锺,然后也回斟一杯与书童吃。书童道:“小的不敢吃,吃了快脸红,只怕爹来看见。”李瓶儿道:“我赏你吃,怕怎的!”于是磕了头起来,一吸而饮之。李瓶儿把各样嘎饭拣在一个碟儿里,教他吃。那小厮一连陪他吃了两大杯,怕脸红就不敢吃,就出来了。到了前边铺子里,还剩了一半点心嘎饭,摆在柜上,又打了两提坛酒,请了傅伙计、贲四、陈敬济、来兴儿、玳安儿。众人都一阵风卷残云,吃了个净光。就忘了教平安儿吃。

那平安儿坐在大门首,把嘴谷都着。不想西门庆约后晌从门外拜了客来家,平安看见也不说。那书童听见喝道之声,慌的收拾不迭,两三步叉到厅上,与西门庆接衣服。西门庆便问:“今日没人来?”书童道:“没人。”西门庆脱了衣服,摘去冠帽,带上巾帻,走到书房内坐下。书童儿取了一盏茶来递上,西门庆呷了一口放下。因见他面带红色,便问:“你那里吃酒来?”这书童就向桌上砚台下取出一纸柬帖与西门庆瞧,说道:“此是后边六娘叫小的到房里,与小的的,说是花大舅那里送来,说车淡等事。六娘教小的收着与爹瞧。因赏了小的一盏酒吃,不想脸就红了。”西门庆把帖观看,上写道:“犯人车淡四名,乞青目。”看了,递与书童,分咐:“放在我书箧内,教答应的明日衙门里禀我。”书童一面接了放在书箧内,又走在旁边侍立。西门庆见他吃了酒,脸上透出红白来,红馥馥唇儿,露着一口糯米牙儿,如何不爱。于是淫心辄起,搂在怀里,两个亲嘴咂舌头。那小郎口噙香茶桂花饼,身上薰的喷鼻香。西门庆用手撩起他衣服,褪了花裤儿,摸弄他屁股。因嘱咐他:“少要吃酒,只怕糟了脸。”书童道:“爹分咐,小的知道。”两个在屋里正做一处。忽一个青衣人,骑了一匹马,走到大门首,跳下马来,向守门的平安作揖,问道:“这里是问刑的西门庆老爹家?”那平安儿因书童不请他吃东道,把嘴头子撅着,正没好气,半日不答应。那人只顾立着,说道:“我是帅府周老爷差来,送转帖与西门老爹看。明日与新平寨坐营须老爹送行,在永福寺摆酒。也有荆都监老爹,掌刑夏老爹,营里张老爹,每位分资一两。迳来报知,累门上哥禀禀进去,小人还等回话。”那平安方拿了他的转帖入后边,打听西门庆在花园书房内,走到里面,转过松墙,只见画童儿在窗外台基上坐的,见了平安摆手儿。那平安就知西门庆与书童干那不急的事,悄悄走在窗下听觑。半日,听见里边气呼呼,跐的地平一片声响。西门庆叫道:“我的儿,把身子调正着,休要动。”就半日没听见动静。只见书童出来,与西门庆舀水洗手,看见平安儿、画童儿在窗子下站立,把脸飞红了,往后边拿去了。平安拿转帖进去,西门庆看了,取笔画了知,分咐:“后边问你二娘讨一两银子,教你姐夫封了,付与他去。”平安儿应诺去了。

书童拿了水来,西门庆洗毕手,回到李瓶儿房中。李瓶儿便问:“你吃酒?教丫头筛酒你吃。”西门庆看见桌子底下放着一坛金华酒,便问:“是那里的?”李瓶儿不好说是书童儿买进来的,只说:“我一时要想些酒儿吃,旋使小厮街上买了这坛酒来。打开只吃了两锺儿,就懒待吃了。”西门庆道:“阿呀,前头放着酒,你又拿银子买!前日我赊了丁蛮子四十坛河清酒,丢在西厢房内。你要吃时,教小厮拿钥匙取去。”李瓶儿还有头里吃的一碟烧鸭子、一碟鸡肉、一碟鲜鱼没动,教迎春安排了四碟小菜,切了一碟火薰肉,放下桌儿,在房中陪西门庆吃酒。西门庆更不问这嘎饭是那里,可见平日家中受用,这样东西无日不吃。西门庆饮酒中间想起,问李瓶儿:“头里书童拿的那帖儿是你与他的?”李瓶儿道:“是门外花大舅那里来说,教你饶了那伙人罢。”西门庆道:“前日吴大舅来说,我没依。若不是,我定要送问这起光棍。既是他那里分上,我明日到衙门里,每人打他一顿放了罢。”李瓶儿道:“又打他怎的?打的那雌牙露嘴。甚么模样!”西门庆道:“衙门是这等衙门,我管他雌牙不雌牙。还有比他娇贵的。”李瓶儿道:“我的哥哥,你做这刑名官,早晚公门中与人行些方便儿,也是你个阴骘,别的不打紧,只积你这点孩儿罢。”西门庆道:“可说什么哩!”李瓶儿道:“你到明日,也要少拶打人,得将就将就些儿,那里不是积福处。”西门庆道:“公事可惜不的情儿。”

两个正饮酒中间,只见春梅掀帘子进来。见西门庆正和李瓶儿腿压着腿儿吃酒,说道:“你每自在吃的好酒儿!这咱晚就不想使个小厮接接娘去?只有来安儿一个跟着轿子,隔门隔户,只怕来晚了,你倒放心!”西门庆见他花冠不整,云鬓蓬松,便满脸堆笑道:“小油嘴儿,我猜你睡来。”李瓶儿道:“你头上挑线汗巾儿跳上去了,还不往下拉拉!”因让他:“好甜金华酒,你吃锺儿。”西门庆道:“你吃,我使小厮接你娘去。”那春梅一手按着桌儿且兜鞋,因说道:“我才睡起来,心里恶拉拉,懒待吃。”西门庆道:“你看不出来,小油嘴吃好少酒儿!”李瓶儿道:“左右今日你娘不在,你吃上一锺儿怕怎的?”春梅道:“六娘,你老人家自饮,我心里本不待吃,俺娘在家不在家便怎的?就是娘在家,遇着我心不耐烦,他让我,我也不吃。”西门庆道:“你不吃,喝口茶儿罢。我使迎春前头叫个小厮,接你娘去。”因把手中吃的那盏木樨芝麻薰笋泡茶递与他。那春梅似有如无,接在手里,只呷了一口,就放下了。说道:“你不要教迎春叫去。我已叫了平安儿在这里,他还大些。”西门庆隔窗就叫平安儿。那小厮应道:“小的在这里伺候。”西门庆道:“你去了,谁看大门?”平安道:“小的委付棋童儿在门上。”西门庆道:“既如此,你快拿个灯笼接去罢。”

平安儿于是迳拿了灯笼来迎接潘金莲。迎到半路,只见来安儿跟着轿子从南来了。原来两个是熟抬轿的,一个叫张川儿,一个叫魏聪儿。走向前一把手拉住轿扛子,说道:“小的来接娘来了。”金莲就叫平安儿问道:“是你爹使你来接我?谁使你来?”平安道:“是爹使我来倒少!是姐使了小的接娘来了。”金莲道:“你爹想必衙门里没来家。”平安道:“没来家?门外拜了人,从后晌就来家了。在六娘房里,吃的好酒儿。若不是姐旋叫了小的进去,催逼着拿灯笼来接娘,还早哩!小的见来安一个跟着轿子,又小,只怕来晚了,路上不方便,须得个大的儿来接才好,小的才来了。”金莲又问:“你来时,你爹在那里?”平安道:“小的来时,爹还在六娘房里吃酒哩。姐禀问了爹,才打发了小的来了。”金莲听了,在轿子内半日没言语,冷笑骂道:“贼强人,把我只当亡故了的一般。一发在那淫妇屋里睡了长觉罢了。到明日,只交长远倚逞那尿胞种,只休要晌午错了。张川儿在这里听着,也没别人。你脚踏千家门、万家户,那里一个才尿出来的孩子,拿整绫缎尺头裁衣裳与他穿?你家就是王十万,使的使不的?”张川儿接过来道:“你老人家不说,小的也不敢说,这个可是使不的。不说可惜,倒只恐折了他,花麻痘疹还没见,好容易就能养活的大?去年东门外一个大庄屯人家,老儿六十岁,见居着祖父的前程,手里无碑记的银子,可是说的牛马成群,米粮无数,丫鬟侍妾成群,穿袍儿的身边也有十七八个。要个儿子花看样儿也没有。东庙里打斋,西寺里修供,舍经施像,那里没求到?不想他第七个房里,生了个儿子,喜欢的了不得。也像咱当家的一般,成日如同掌儿上看擎,锦绣窝儿里抱大。糊了三间雪洞儿的房,买了四五个养娘扶持。成日见了风也怎的,那消三岁,因出痘疹丢了。休怪小的说,倒是泼丢泼养的还好。”金莲道:“泼丢泼养?恨不得成日金子儿裹着他哩!”平安道:“小的还有桩事对娘说。小的若不说,到明日娘打听出来,又说小的不是了。便是韩伙计说的那伙人,爹衙门里都夹打了,收在监里,要送问他。今早应二爹来和书童儿说话,想必受了几两银子,大包子拿到铺子里,就便凿了二三两使了。买了许多东西嘎饭,在来兴屋里,教他媳妇子整治了,掇到六娘屋里,又买了两瓶金华酒,先和六娘吃了。又走到前边铺子里,和傅二叔、贲四、姐夫、玳安、来兴众人打伙儿,直吃到爹来家时分才散了。”金莲道:“他就不让你吃些?”平安道:“他让小的?好不大胆的蛮奴才!把娘每还不放在心上。不该小的说,还是爹惯了他,爹先不先和他在书房里干的龌龊营生。况他在县里当过门子,什么事儿不知道?爹若不早把那蛮奴才打发了,到明日咱这一家子吃他弄的坏了。”金莲问道:“在你六娘屋里吃酒,吃的多大回?”平安儿道:“吃了好一日儿。小的看见他吃的脸儿通红才出来。”金莲道:“你爹来家,就不说一句儿?”平安道:“爹也打牙粘住了,说什么!”金莲骂道:“恁贼没廉耻的昏君强盗!卖了儿子招女婿,彼此腾倒着做。”嘱付平安:“等他再和那蛮奴才在那里干这龌龊营生,你就来告我说。”平安道:“娘分咐,小的知道。娘也只放在心里,休要题出小的一字儿来。”于是跟着轿子,直说到家门首。

潘金莲下了轿,先进到后边拜见月娘。月娘道:“你住一夜,慌的就来了?”金莲道:“俺娘要留我住。他又招了俺姨那里一个十二岁的女孩儿在家过活,都挤在一个炕上,谁住他!又恐怕隔门隔户的,教我就来了。俺娘多多上复姐姐:多谢重礼。”于是拜毕月娘,又到李娇儿、孟玉楼众人房里,都拜了。回到前边,打听西门庆在李瓶儿屋里说话,迳来拜李瓶儿。李瓶儿见他进来,连忙起身,笑着迎接进房里来,说道:“姐姐来家早,请坐,吃锺酒儿。”教迎春:“快拿座儿与你五娘坐。”金莲道:“今日我偏了杯,重复吃了双席儿,不坐了。”说着,扬长抽身就去了。西门庆道:“好奴才,恁大胆,来家就不拜我拜儿?”那金莲接过来道:“我拜你?还没修福来哩。奴才不大胆,什么人大胆!”看官听说:潘金莲这几句话,分明讥讽李瓶儿,说他先和书童儿吃酒,然后又陪西门庆,岂不是双席儿,那西门庆怎晓得就理。正是:

\[
情知语是针和丝,就地引起是非来。
\]
