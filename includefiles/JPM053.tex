%# -*- coding:utf-8 -*-
%%%%%%%%%%%%%%%%%%%%%%%%%%%%%%%%%%%%%%%%%%%%%%%%%%%%%%%%%%%%%%%%%%%%%%%%%%%%%%%%%%%%%


\chapter{潘金莲惊散幽欢\KG 吴月娘拜求子息}


词曰:

\[
小院闲阶玉砌,墙隈半簇兰芽。一庭萱草石榴花,多子宜男爱插。休使风吹雨打,老天好为藏遮。莫教变作杜鹃花,粉褪红销香罢。
\]

话说陈敬济与金莲不曾得手,怅怏不题。单表西门庆赴黄、安二主事之席。乘着马,跟随着书童、玳安四五人,来到刘太监庄上。早有承局报知,黄、安二主事忙整衣冠,出来迎接。那刘太监是地主,也同来相迎。西门庆下了马,刘太监一手挽了西门庆,笑道:“咱三个等候的好半日了,老丈却才到来。”西门庆答道:“蒙两位老先生见招,本该早来,实为家下有些小事,反劳老公公久待,望乞恕罪。”三个大打恭,进仪门来。让到厅上,西门庆先与黄主事作揖,次与安主事、刘太监都作了揖,四人分宾主而坐。第一位让西门庆坐了,第二就该刘太监坐。刘太监再四不肯,道:“咱忝是房主,还该两位老先生,是远客。”安主事道:“定是老先儿。”西门庆道:“若是序齿,还该刘公公。”刘大监推却不过,向黄、安两主事道:“斗胆占了。”便坐了第二位。黄、安二主事坐了主席。一班小优儿上来磕了头,左右献过茶,当值的就递上酒来。黄、安二主事起身安席坐下。小优儿拿檀板、琵琶、弦索、箫管上来,合定腔调,细细唱了一套《宜春令》“青阳候烟雨淋”。唱毕,刘太监举杯劝众官饮酒。安主事道:“这一套曲儿,做的清丽无比,定是一个绝代才子。况唱的声音嘹亮,响遏行云,却不是个双绝了么!”西门庆道:“那个也不当奇,今日有黄、安二位做了贤主,刘公公做了地主,这才是难得哩!”黄主事笑道:“也不为奇。刘公公是出入紫禁,日觐龙颜,可不是贵臣?西门老丈,堆金积玉,仿佛陶朱,可不是富人?富贵双美,这才是奇哩!”四个人哈哈大笑。当值的斟上酒来,又饮了一回。小优儿又拿碧玉洞箫,吹得悠悠咽咽,和着板眼,唱一套《沽美酒》“桃花溪,杨柳腰”的时曲。唱毕,众客又赞了一番,欢乐饮酒不题。

且说陈敬济因与金莲不曾得手,耐不住满身欲火。见西门庆吃酒到晚还未来家,依旧闪入卷棚后面,探头探脑张看。原来金莲被敬济鬼混了一场,也十分难熬,正在无人处手托香腮,沉吟思想。不料敬济三不知走来,黑影子里看见了,恨不的一碗水咽将下去。就大着胆,悄悄走到背后,将金莲双手抱住,便亲了个嘴,说道:“我前世的娘!起先吃孟三儿那冤儿打开了,几乎把我急杀了。”金莲不提防,吃了一吓。回头看见是敬济,心中又惊又喜,便骂道:“贼短命,闪了我一闪,快放手,有人来撞见怎了!”敬济那里肯放,便用手去解他裤带。金莲犹半推半就,早被敬济一扯扯断了。金莲故意失惊道:“怪贼囚,好大胆!就这等容容易易要奈何小丈母!”敬济再三央求道:“我那前世的亲娘,要敬济的心肝煮汤吃,我也肯割出来。没奈何,只要今番成就成就。”敬济口里说着,腰下那话已是硬帮帮的露出来,朝着金莲单裙只顾乱插。金莲桃颊红潮,情动久了。初还假做不肯,及被敬济累垂敖曹触着,就禁不的把手去摸。敬济便趁势一手掀开金莲裙子,尽力往内一插,不觉没头露脑。原来金莲被缠了一回,臊水湿漉漉的,因此不费力送进了。两个紧傍在红栏干上,任意抽送,敬济还嫌不得到根,教金莲倒在地下:“待我奉承你一个不亦乐乎!”金莲恐散了头发,又怕人来,推道:“今番且将就些,后次再得相聚,凭你便了。”一个“达达”连声,一个“亲亲”不住,厮併了半个时辰。只听得隔墙外籁籁的响,又有人说话,两个一哄而散。

敬济云情未已,金莲雨意方浓。却是书童、玳安拿着冠带拜匣,都醉醺醺的嚷进门来。月娘听见,知道是西门庆来家,忙差小玉出来看。书童、玳安道:“爹随后就到了。我两人怕晚了,先来了。”不多时,西门庆下马进门,已醉了,直奔到月娘房里来。搂住月娘就待上床。月娘因要他明日进房,应二十三壬子日服药行事,便不留他,道:“今日我身子不好,你往别房里去罢。”西门庆笑道:“我知道你嫌我醉了,不留我。也罢,别要惹你嫌。我去了,明晚来罢。”月娘笑道:“我真有些不好,月经还未净。谁嫌你?明晚来罢。”西门庆就往潘金莲房里去了。金莲正与敬济不尽兴回房,眠在炕上,一见西门庆进来,忙起来笑迎道:“今日吃酒,这咱时才来家。”西门庆也不答应,一手搂将过来,连亲了几个嘴,一手就下边一摸,摸着他牝户,道:“怪小淫妇儿,你想着谁来?兀那话湿搭搭的。”金莲自觉心虚,也不做声。只笑推开了西门庆,向后边澡牝去了。当晚与西门庆云情雨意,不消说得。

且表吴月娘次日起身,正是二十三壬子日,梳洗毕,就教小玉摆着香桌,上边放着宝炉,烧起名香,又放上《白衣观音经》一卷。月娘向西皈依礼拜,拈香毕,将经展开,念一遍,拜一拜,念了二十四遍,拜了二十四拜,圆满。然后箱内取出丸药放在桌上,又拜了四拜,祷告道:“我吴氏上靠皇天,下赖薛师父、王师父这药,仰祈保佑,早生子嗣。”告毕,小玉烫的热酒,倾在盏内。月娘接过酒盏,一手取药调匀,西向跪倒,先将丸药咽下,又取末药也服了,喉咙内微觉有些腥气。月娘迸着气一口呷下,又拜了四拜。当日不出房,只在房里坐的。

西门庆在潘金莲房中起身,就叫书童写谢宴贴,往黄、安二主事家谢宴。书童去了,就是应伯爵来到。西门庆出来,应伯爵作了揖,说道:“哥,昨在刘太监家吃酒,几时来家?”西门庆道:“承两公十分相爱,灌了好几杯酒,归路又远,更余来家。已是醉了,这咱才起身。”玳安捧出早饭,西门庆正和伯爵同吃,又报黄主事、安主事来拜。西门庆整衣冠,教收过家活出迎。应伯爵忙回避了。黄、安二主事一齐下轿。进门厮见毕,三人坐下,一面捧出茶来吃了。黄、安二主事道:“夜来有亵,”西门庆道:“多感厚情,正要叩谢两位老先生,如何反劳台驾先施!”安主事道:“昨晚老先生还未尽兴,为何就别了?”西门庆道:“晚生已大醉了。临起身,又被刘公公灌上十数杯葡萄酒,在马上就要呕,耐得到家,睡到今日还有些不醒哩。”笑了一番,又吃过三杯茶,说些闲话,作别去了。应伯爵也推事故家去。西门庆回进后边吃了饭,就坐轿答拜黄、安二主事去。又写两个红礼帖,吩咐玳安备办两副下程,赶到他家面送。当日无话。

西门庆来家,吴月娘打点床帐,等候进房。西门庆进了房,月娘就教小玉整设肴馔,烫酒上来,两人促膝而坐。西门庆道:“我昨夜有了杯酒,你便不肯留我,又假推甚么身子不好,这咱捣鬼!”月娘道,“这不是捣鬼,果然有些不好。难道夫妻之间恁地疑心?”西门庆吃了十数杯酒,又吃了些鲜鱼鸭腊,便不吃了,月娘交收过了。小玉熏的被窝香喷喷的,两个洗澡已毕,脱衣上床。枕上绸缪,被中缱绻,言不可尽。这也是吴月娘该有喜事,恰遇月经转,两下似水如鱼,便得了子了。正是:

\[
花有并头莲并蒂,带宜同挽结同心。
\]

次日,西门庆起身梳洗,月娘备有羊羔美酒、鸡子腰子补肾之物,与他吃了,打发进衙门去。西门庆衙门散了回来,就进李瓶儿房看哥儿。李瓶儿抱着孩子向西门庆道:“前日我有些心愿未曾了。这两日身子有些不好,坐净桶时,常有些血水淋得慌。早晚要酬酬心愿,你又忙碌碌的,不得个闲空。”西门庆道:“你既要了愿时,我叫玳安去接王姑子来,与他商量,做些好事就是了。”便叫玳安,吩咐接王姑子。玳安应诺去了。

书童又报:“常二叔和应二爹来到。”西门庆便出迎厮见。应伯爵道:“前日谢子纯在这里吃酒,我说的黄四、李三的那事,哥应付了他罢。”西门庆道:“我那里有银子?”应伯爵道:“哥前日已是许下了,如何又变了卦?哥不要瞒我,等地财主,说个无银出来?随分凑些与他罢。”西门庆不答应他,只顾呆了脸看常峙节。常峙节道:“连日不曾来,哥,小哥儿长养么?”西门庆道:“生受注念,却才你李家嫂子要酬心愿,只得去请王姑子来家做些好事。”应伯爵道:“但凡人家富贵,专待子孙掌管。养得来时,须要十分保护。譬如种五谷的,初长时也得时时灌溉,才望个秋收。小哥儿万金之躯,是个掌中珠,又比别的不同。小儿郎三岁有关,六岁有厄,九岁有煞,又有出痧出痘等症。哥,不是我口直,论起哥儿,自然该与他做些好事,广种福田。若是嫂子有甚愿心,正宜及早了当,管情交哥儿无灾无害好养。”说话间,只见玳安来回话道:“王姑子不在庵里,到王尚书府中去了。小的又到王尚书府中找寻他,半日才得出来。与他说了,便来了。”西门庆听罢,依旧和伯爵、常峙节说话儿,一处坐地,书童拿些茶来吃了。伯爵因开言道:“小弟蒙哥哥厚爱,一向因寒家房子窄隘,不敢简亵,多有疏失。今日禀明了哥,若明后日得空,望哥同常二哥出门外花园里顽耍一日,少尽兄弟孝顺之心。”常峙节从旁赞道:“应二哥一片献芹之心,哥自然鉴纳,决没有见却的理。”西门庆道:“若论明日,到没事,只不该生受。”伯爵道:“小弟在宅里,筷子也不知吃了多少下去,今日一杯水酒,当的甚么。”西门庆道:“既如此,我便不往别处去了。”伯爵道:“只是还有一件——小优儿,小弟便叫了。但郊外去,必须得两个唱的去,方有兴趣。”西门庆道:“这不打紧,我叫人去叫了吴银儿与韩金钏儿就是了。”伯爵道:“如此可知好哩。只是又要哥费心,不当。”西门庆一面就叫琴童,吩咐去叫吴银儿、韩金钏儿,明日早往门外花园内唱。琴童应诺去了。

不多时,王姑子来到厅上,见西门庆道个问讯:“动问施主,今日见召,不知有何吩咐?老身因王尚书府中有些小事去了,不得便来,方才得脱身。”西门庆道:“因前日养官哥许下些愿心,一向忙碌碌,未曾完得。托赖皇天保护,日渐长大。我第一来要酬报佛恩,第二来要消灾延寿,因此请师父来商议。”王姑子道:“小哥儿万金之躯,全凭佛力保护。老爹不知道,我们佛经上说,人中生有夜叉罗刹,常喜啖人,令人无子,伤胎夺命,皆是诸恶鬼所为。如今小哥儿要做好事,定是看经念佛,其余都不是路了。”西门庆便问做甚功德好,王姑子道:“先拜卷《药师经》,待回向后,再印造两部《陀罗经》,极有功德。”西门庆问道:“不知几时起经?”王姑子道:“明日到是好日,就我庵中完愿罢。”西门庆点着头道:“依你,依你。”

王姑子说毕,就往后边,见吴月娘和六房姊妹都在李瓶儿房里。王姑子各打了问讯。月娘便道:“今日央你做好事保护官哥,你几时起经头?”王姑子道:“来日黄道吉日,就我庵里起经。”小玉拿茶来吃了。李瓶儿因对王姑子道:“师父,我还有句话,一发央及你。”王姑子道:“你老人家有甚话,但说不妨。”李瓶儿道:“自从有了孩子,身子便有些不好。明日疏意里边,带通一句何如?行的去,我另谢你。”王姑子道:“这也何难。且待写疏的时节,一发写上就是了。”正是:

\[
祸因恶积非无种,福自天来定有根。
\]
